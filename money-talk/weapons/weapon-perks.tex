\subsection{Cвойства оружия}
\paragraph{Бронебойное:} оружие ополовинивает Бонус доспеха и щита цели. Обратите внимание, что это свойство не влияет на Прочность цели.
\paragraph{Боеприпасы:} некоторые виды оружия нуждаются в боеприпасах. Без них оно  — всего лишь безделушка(хотя если это что-то тяжелое, им можно орудовать, как импровизированным оружием ближнего боя). СП 10 зарядов для любого вида оружия и 1 заряда для Противотанкового оружия равна \textbf{|СП оружия — 10|}. Стоимость 1 заряда для любого оружия, кроме Противотанкового равна \textbf{|СП оружия — 15|} (минимум 1).
\paragraph{Возврат Х:} при выпадении Х или большего числа во время проверки Мт оружие не только наносит Пв, но и возвращается в руку к владельцу.
\paragraph{Громоздкое:} используется с Помехой в тесных помещениях и густых зарослях. Если оружие одновременно и Громоздкое, и Длинное, герой получает 2 Помехи. Герой не может использовать Громоздкое оружие лежа для атак в ближнем бою.
\paragraph{Двуручное:} требует 2 рук для использования.
\paragraph{Дистанция X/Y:} Оружие с этим свойством является дальнобойным и не предназначено для совершения атак в ближнем бою. \textbf{X} - Ближняя Дистанция оружия, \textbf{Y} - Дальняя Дистанция оружия. В графе БПв для этго оружия через черту написан урон для Ближней и Дальней дистанции соответственно.
\paragraph{Длинное:} позволяет атаковать врага в 2 метрах от героя. Цели, находящиеся ближе, герой атакует с Помехой. Длинное оружие используется с Помехой в тесных помещениях, густых зарослях и других подобных условиях.
\paragraph{Кавалерийское:} удвойте успешно нанесенные Пв, если геройверхом и применяет Атаку с разбега.
\paragraph{Кастет:} позволяет использовать как Навык Рукопашного боя, так и Навык Владения оружием в формуле Дб.
\paragraph{Класс повреждений Х(КП Х):} Добавляет значение Х к Размеру оружия для определения условий Подавляющего превосходства, не меняя размер оружия для определения других его свойств.

\paragraph{Кувалда:} если нанесенные Пв превышают МСл атакованного существа, оно падает. 
\newline
Маленькие существа падают, если нанесенные Пв превышают 1/2 их МСл, Крошечные — падают, если нанесенные Пв превышают 1/4 их МСл.
\newline
Большие существа падают, если нанесенные Пв превышают их МСл более чем в 2 раза, Огромные — если нанесенные Пв превышают их МСл более чем в 3 раза, Громадные — если нанесенные Пв превышают их МСл более чем в 4 раза.
\newline
Если оружие ближнего боя с этим свойством используется в одной руке, увеличьте необходимые для падения Пв в 2 раза. То есть для того, чтобы сбить с ног существо Среднего размера, понадобится превысить его МСл более чем в 2 раза.
\paragraph{Легкое:} позволяет совершать серии молниеносных выпадов и эффективно атаковать с оружием в каждой руке.
\paragraph{Тип оружия:} Любое дальнобойное оружие нуждается в боеприпасах. Но благодаря унификации герою не нужно носить разные боеприпасы для \textit{каждой} еденицы оружия, которая есть у него в арсенале. Стоимость и свойства боеприпасов описаны в следующем разделе. Тип оружия определяет тип боеприпасов, который использует это оружие:
\begin{center}
\begin{tabular}{|c|p{3cm}|p{10cm}|}
\hline
Тип & Описание & Описание боеприпасов\\ \hline
Р & револьверы и крупнокалиберные пистолеты & медленные пули, которые обычно легко купить\\ \hline
В & винтовки & быстрые пули с хорошей пробивающей способностью\\ \hline
Д & дробовики & дробь и картечь. Реже - пули\\ \hline
О & пулеметы и крупнокалиберные орудия & действительно большие калибры. Размеры настолько высоки, что в них можно поместить взрывчатку\\ \hline
Г & гранатометы и ракетные установки & Унифицированные гранаты и ракеты. Поражающий эффект зависит от начинки\\ \hline
Э & энергетическое оружие & Батареи всех пород и мастей. Если у оружия класса Э нет свойства Потребления, оно тратит 1 Заряд за выстрел.\\ \hline
Б\textsuperscript{ф} & кинетические ускорители & Металические болты с сверхтвердым сердечником. Дешевы в изготовлении, но если их хорошенько разогнать, становятся крайне смертоносны.\\ \hline
М & метательное & Оружие использует метательный боеприпас и приводится в действие силой стрелка. \\ \hline
У & уникальный боеприпас & Оружие использует не стандартизированный боеприпас, который подойдет только для него. СП 10 зарядов для такого вида оружия равна \textbf{|СП оружия — 10|}. Стоимость 1 заряда для этого оружия равна \textbf{|СП оружия — 12|} (минимум 1). \\ \hline
\end{tabular}
\end{center}
\paragraph{Магазин/Скорострельность(МСк) Х/Y:} дальнобойное оружие с этим свойством способно выпускать множество снарядов сразу или один за другим и не требовать Перезарядки сразу после первого выстрела. \textbf{X} количество выстрелов, которое может сделать оружие без Перезарядки, а \textbf{Y} — число выстрелов, которое герой может произвести за время своего Действия. Скорострельное оружие может использоваться двумя способами — для поражения одной цели или нескольких. При использовании скорострельного оружия герой может одновременно поразить число целей, не превышающее его \textbf{|ММд+1|} (минимум 2 цели).
\newline
Если Скорострельное оружие используется против одной цели, повысьте БПв оружия на 1 за каждый снаряд после первого, выпущенный по ней. Например, скорострельный арбалет имеет БПв +1 и Скорострельность 3. Если герой трижды стреляет из скорострельного арбалета в одну цель, БПв скорострельного арбалета возрастает до +3 (то есть на 2).
Герой не может совершать Быструю атаку скорострельным оружием, но может использовать место этого Беглый огонь. В разделе Маневры есть полное описание этого маневра.
\begin{tcolorbox}
Если у оружие есть свойство Магазин или Потребление, то это значит, что у оружия есть свойство Перезарядка, даже если это явно не указано. Оружие с МСк надо Перезаряжать, когда закончатся заряды в магазине, а оружие с Потреблением - когда опустошена Энергоячейка.
\end{tcolorbox}
\paragraph{Тип/Магазин/Скорострельность(ТМС):} объедененный столбец, в котором через черту отражены класс оружия, размер магазина и его скорострельность.
\paragraph{Метательное:} может быть использовано и в ближнем и в дальнем бою. При метании оружие обычно падает в область, в которой находилась цель. Оно может быть поднято и использовано повторно. При Дистанционной атаке Метательным оружием герой добавляет свой МСл к БПв оружия. У метательного оружия \textbf{нет} свойства Перезарядка - количество выстрелов в раунд ограничена только Скорострельностью оружия. Если герой держит в каждой руке два одинаковых Метательных оружия, он может метнуть оба в рамках маневра Беглый огонь или Атака, повысив скорострельность оружия на 1, но получив при этом Помеху на этом маневр и дополнительно вторую Помеху, если оружие Громоздкое. Если у героя есть трюк Амбидекстер, он получает на одну Помеху меньше при совершении этого маневра. В таблицах свойство Метательное указано в столбце ТМС как тип оружия.
\begin{tcolorbox}
Используя метательное оружие с дополнительным свойством \textbf{Снаряды} герой не метает оружие целиком, а использует подготовленные для него снаряды. Им нельзя производить атаку в Ближнего боя, однако стоимость выстрела гораздо ниже, когда во врага летит только часть оружия. СП 10 зарядов для такого вида оружия равна \textbf{|СП оружия — 10|}. Стоимость 1 заряда для этого оружия равна \textbf{|СП оружия — 12|} (минимум 1).
\end{tcolorbox}
\paragraph{Огнемет:} при атаке огнемет поражает все объекты, находящиеся на линии между стрелком и целью атаки. Совершите только одну проверку Меткости и определите повреждения для всех пораженных объектов, исходя из нее. В случае выпадения КУ его эффекты применяются ко всем объектам, пораженным огнеметом.
\paragraph{Отдача:} герой должен отказаться от Перемещения, если желает выпустить из оружия с этим свойством 5 или больше зарядов.
\paragraph{Перезарядка:} для перезарядки оружия с этим свойством герой должен отказаться от Действия или Перемещения. Оружие не может использоваться при Быстрой атаке. Арбалеты и пороховое оружие требуют 2 свободных рук при перезарядке.
\paragraph{Потайное:} герой получает Преимущество на проверки Скрытности и Ловкости рук при попытках спрятать оружие.
\paragraph{Потребление Х:} Оружие с этим свойством вместо боеприпасов использует энергию и тратит Х Зарядов за 1 выстрел или удар.
%\paragraph{Противотанковое:} оружие предназначено для поражения тяжелобронированных целей. У остальных не так уж много шансов пережить попадание из него. При произведении выстрела из противотанкового оружия, \troubleControl Меткости против ЗЩ цели.
%\trouble
%{В яблочко}%no sweat name
%{Цель зацепило снарядом. Она теряет все свои ЕЗ.}%no sweat description
%{Прямое попадание}%tough day name
%{Цель слегка задело. Она теряет 1/2 своих ЕЗ.}%tough day description
%{Попадание}%we have trouble name
%{Цель задело осколками или взрывом. Она теряет 1/4 своих ЕЗ.}%we have trouble description
%{Промах}%fiasco name
%{Цель обдало грязью, травой, мелкими камушками (и, возможно, останками тех, кому повезло меньше), но в остальном она невредима}%fiasco description
\paragraph{Пулемет:} оружие не приспособлено для одиночных выстрелов. Если герой использует оружие для поражения нескольких целей, то должен выпустить минимум 3 пули в каждую.
\paragraph{Снайперское:} при стрельбе на дальнюю дистанцию оружие игнорирует штрафы Зон поражения, если герой проводит Прицеливаясь хотя бы один полный Круг. При стрельбе на ближней дистанции герой получает Помеху.
\paragraph{Сошки(с):} оружие оснащено сошками для стрельбы с упора. Если носитель имеет возможность установить сошки на какую-либо поверхность, используйте значение Минимальной силы с пометкой (с), в противном случае используйте значение после черты. Если у оружия нет значения Минимальной силы без символа (с), значит, стрельба с рук невозможна.
\paragraph{Тип Повреждений:}
в зависимости от типа, повреждения могут иметь разные эффекты КУ и могут быть увеличены или уменьшены в зависимости от того, какие есть сопротивления или уязвимости у цели. Атаки имеют один из следующих типов Повреждений:
\newline
\textbf{(Д)}робящие, \textbf{(Е)}дкие, \textbf{(К)}олющие, \textbf{(Л)}едяные, \textbf{(О)}гненные, \textbf{(П)}роникающие, \textbf{(Р)}убящие, \textbf{(Э)}лектрические, \textbf{(Я)}довитые.
\newline
Если тип повреждений отмечен *, то он зависит от заряженных в оружие боеприпасов. См. раздел боеприпасов для определения типа повреждений.
\newline
Если оружие имеет несколько типов повреждений через черту(К/Р или Р/Д), герой может выбирать, какой тип повреждений наносить при атаке. Если типы повреждений перечислены слитно(ОЭ или РЯ), то оружие наносит одновременно несколько типов повреждений. Подробнее об этом написано в главе Боевые Столкновения.
\paragraph{Удавка:} этим свойством обладает любое оружие, достаточно гибкое, чтобы захлестнуть конечности противника. Для применения свойства необходимо две руки. Оружие может использоваться для Захватов. Удвойте Пв, успешно нанесенные Удавкой в шею. Существа, чьи ЕЗ опустились до 0 в результате Пв, нанесенных Удавкой в шею, теряют сознание, а не умирают, если атакующий того пожелает. Существа, у которых БДЗщ (но не БЩЗщ) составляет 6 или больше, не могут быть атакованы Удавкой в шею — она надежно защищена.
\paragraph{Универсальное:} может использоваться в 1 или 2 руках. Смена хвата является Быстрым действием. В БПв и тСл указаны для одной и двух рук через черту.
\paragraph{Упредительный удар:} если противник входит в зону поражения оружия, герой может потратить Быстрое действие и провести внеочередную атаку по нападающему или его скакуну. Выход из зоны действия оружия и передвижение в ней не провоцируют Упредительный удар.
\paragraph{Фехтовальное:} герой может заменить МСл на ММд при подсчете Дб.
\paragraph{Хрупкое:} из-за особенностей конструкции оружие уязвимо к самым незначительным повреждениям и не предназначено для парирования и блокирования. Ополовиньте ЕЗ оружия(минимум 1 ХП).
\paragraph{Цеп:} игнорирует БЩЗщ, хотя может заявлять щит как область поражения.
\paragraph{Символ «*»} обозначает качества оружия, не указанные в таблице, не входящие в унифицированный перечень Свойств и перечисленные в описательном блоке. \textbf{Символ «Ф»} обозначает оружие, уместное лишь в фантастическом антураже.