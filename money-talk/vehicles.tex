\section{Транспорт}
В этом разделе описаны механизмы и живые существа, которых можно использовать, как транспорт
\newline Символ «Ф» обозначает вид транспорта, уместный лишь в фантастическом антураже.
\newline Если в столбце размера указана литера \textbf{Ж}, то этот вид транспорта является животным.
\paragraph{Ограничение Модификатора Ловкости(оМЛв):} Крупные транспортные средства ворочаются неохотно, а разогнавшись, не спешат тормозить оМЛв определяет максимальны МЛв, который водитель может добавить к проверкам Эксплуатации при управлении транспортным средством.
\paragraph{Ограничение Модификатора Ловкости(оМЛв) для животных:} Не вся животина резво подчиняется командам погонщика или наездника. Максимальный МЛв, который может герой добавить к проверкам Общения с животными равен МЛв самого животного.
\paragraph{Проходимость транспортного средства(П):} транспорт не может передвигаться по местности с Опасностью, превышающей его Проходимость. Проходимость животных увеличивается на 1, если они не несут груза или всадников.
\begin{itemize}
\item литерой \textbf{Л} обозначена способность транспорта к полетам. Проходимость летающих транспортных средств учитывается только при взлете и посадке. Также они всегда могут передвигаться на пределе скорости, если возникла такая необходимость.
\item литерой \textbf{В} обозначено то, что транспорт может перемещаться только по воде.
\item литерой \textbf{К} обозначена способность транспорта к совершению межпланетных и межзвездных путешествий. \tbd(описать, что скорость межпланетного перемещения определяется сеттингом, а скорость в таблице определяет скорость при околопланетном маневрировании)
\end{itemize}
\paragraph{Защита транспортного средства:} предполагается, что транспортные средства из этой уже оснащены всеми возможными защитными приспособлениями в пределах разумной целесообразности. Модификатор Ловкости водителя прибавляется (или отнимается, если он отрицательный) к Защите ТС.
\paragraph{Перегруженный транспорт:} если нагрузка транспортного средства превышает Грузоподъемность не более чем в 2 раза, водитель получает Помеху на проверки Эксплуатации. Если нагрузка транспортного средства превышает Грузоподъемность не более чем в 3 раза, водитель получает 2 Помехи на Эксплуатацию.
\paragraph{Буксировка:} предполагается, что транспортное средство может буксировать вес, не превышающий свой собственный. Если буксируемый вес превышает вес транспортного средства не более чем в 2 раза, водитель получает Помеху на Эксплуатацию. Если буксируемый вес превышает вес транспортного средства не более чем в 3 раза, водитель получает 2 Помехи на Эксплуатацию.
\paragraph{Маневренность.} Во многих ситуациях важны не только проходимость и скорость автомобиля, но и его способность избегать препятствий. Маневренность транспорта равна \textbf{|Проходимость — Модификатор размера транспортного средства|}.Маневренность добавляется к навыку
водителя (или отнимается, если она отрицательная) при проверке Эксплуатации для совершения сложных поворотов и перестроений.
\paragraph{Ховер:} в научно-фантастических сеттингах транспортное средство может быть оснащено ховер-левитатором. Его СП возрастает на 5, его Проходимость возрастает до 5.
\paragraph{Потеря ЕЗ транспортным средством:} после потери \textbf{1/3 ЕЗ}
транспортное средство теряет половину своей Скорости. После
потери \textbf{2/3 ЕЗ} проверки Эксплуатации совершаются с Помехой.
\newline
Если транспортное средство одномоментно теряет \textbf{1/4 или более своих ЕЗ}, водитель должен совершить проверку Эксплуатации против 15. При провале транспортное средство глохнет.
\newline
Попадания по двигателю или ходовой части транспортного средства могут быть весьма опасны! Если зона поражения одномоментно теряет \textbf{1/5 или более от максимальных Единиц Здоровья}, она выходит из строя. Лишние Повреждения теряются, но водитель должен совершить проверку Эксплуатации против 15. При провале транспортное средство глохнет.
\newline{Ремонт транспортного средства:} восстановление каждых 5 ЕЗ требует 1 часа работы, наличия Ремонта у механика и имеет СП 1.

\paragraph{Cкорость(Ск)} Скорость трансопорта во время путешествий(в км/ч). Численно она равна Ск транспорта во время Боевых Сцен и она \textbf{значительно ниже} максимальной скорости, которую можно выжать из этого транспорта. Если у транспорта в этой графе присутствует литера М, это значит, что его обычная скорость равна максимальной и этот транспорт не может двигаться \textbf{Маршем}(если только скорость самого медленного члена каравана не в два раза меньше, чем скорость этого транспорта).
\paragraph{Расход топлива(Р)} определяет сколько Зарядов Топлива(или килограмм фуража для животных) тратит тот или иной вид транспорта не каждые 10 км пути.
\paragraph{Грузоподъемность/Вес(ГВ)} определяет количество полезной нагрузки(за исключением полных баков топлива), которую может вести транспорт без перегрузки и вес самого транспорта для определения возможностей буксировки.

\subsection{Описание транспорта}
\genAndGet{transport}{transport}
\endinput
\begin{center}
\begin{longtable}{|p{2.5cm}||c|c|c|c||c|c|c||c|c|c|}
\hline
Название & 
Размер & Зщ & Прч & ЕЗ & 
оМЛв & П & Ск & 
ГВ & Р & СП \\ \hline \hline

Велосипед & 
М & 10 & 1 & 15 & 
- & 4 & СЛ+ВН водителя & 
80/15 & - & 7 \\ \hline

Мотоблок & 
М & 10 & 2 & 15 & 
+3 & 8М & 5* & 
500/50 & 6 & 10 \\ \hline

Мотороллер & 
М & 11 & 2 & 25 & 
- & 3 & 30 & 
100/100 & 3 & 10 \\ \hline

Мопед & 
М & 10 & 1 & 20 & 
- & 4 & 40 & 
100/100 & 3 & 10 \\ \hline

Скутер & 
С & 13 & 2 & 40 & 
+5 & 3 & 60 & 
150/300 & 6 & 12 \\ \hline

Мотоцикл & 
С & 12 & 3 & 40 & 
+5 & 3 & 90 & 
150/300 & 6 & 12 \\ \hline

Мусорный багги & 
Б & 14 & 3 & 50 & 
+4 & 4 & 80 & 
350/750 & 12 & 15 \\ \hline

Малолитражка & 
Б & 13 & 4 & 55 & 
+3 & 2 & 70 & 
420/750 & 9 & 15 \\ \hline

Легковой автомобиль & 
О & 14 & 4 & 60 & 
+3 & 2 & 100 & 
500/1300 & 15 & 20 \\ \hline

Прокачанная тачка & 
О & 13 & 5 & 70 & 
+2 & 2 & 130 & 
420/1800 & 30 & 25 \\ \hline

Спорткар & 
О & 12 & 4 & 65 & 
+4 & 1 & 150 & 
300/1400 & 36 & 25 \\ \hline

Лимузин & 
О & 13 & 7 & 80 & 
+1 & 1 & 100 & 
700/200 & 21 & 20 \\ \hline

Минивэн & 
О & 13 & 5 & 75 & 
+2 & 2 & 80 & 
1000/2000 & 18 & 25 \\ \hline

Джип & 
Б & 14 & 5 & 60 & 
+3 & 3 & 90 & 
550/1700 & 24 & 25 \\ \hline

Арммейский джип & 
О & 14 & 10 & 80 & 
+1 & 4 & 60 & 
1000/2900 & 33 & 30 \\ \hline

Пикап & 
О & 13 & 7 & 70 & 
+2 & 3 & 80 & 
720/2100 & 30 & 25 \\ \hline

Автобус & 
Г & 13 & 7 & 100 & 
+1 & 3 & 50 & 
9500/6000 & 30 & 30 \\ \hline

Грузовик & 
Г & 15 & 10 & 90 & 
+1 & 3 & 50 & 
9000/8000 & 45 & 30 \\ \hline

Армейский грузовик & 
Г & 15 & 13 & 100 & 
+1 & 4 & 50 & 
8000/8000 & 45 & 33 \\ \hline

Тягач & 
Г & 14 & 10 & 110 & 
+1 & 3 & 80 & 
8000/15000* & 45 & 35 \\ \hline

Армейский тягач & 
Г & 14 & 15 & 120 & 
+1 & 4 & 80 & 
10000/34000* & 60 & 38 \\ \hline

Трактор & 
Г & 14 & 10 & 90 & 
+1 & 4 & 20 & 
3500/3500 & 30 & 30 \\ \hline

Бульдозер & 
Г & 12 & 13 & 120 & 
0 & 3 & 8 & 
4000/18000 & 36 & 30 \\ \hline

БМП & 
Г & 17 & 15 & 130 & 
0 & 4 & 40 & 
9000/13000 & 45 & 40 \\ \hline

Танк & 
Г & 17 & 20 & 130 & 
0 & 4 & 30 & 
10000/45000 & 60 & 50 \\ \hline

Легкий винтовой самолет & 
Г & 14 & 5 & 60 & 
- & 1** & 160 & 
400/750 & 45 & 35 \\ \hline

Грузовой винтовой самолет & 
Г & 13 & 12 & 120 & 
0 & 1** & 300 & 
60000/120000 & 90 & 70 \\ \hline

Военный вертолет & 
Г & 17 & 7 & 80 & 
+4 & 4** & 200 & 
5000/7000 & 60 & 50 \\ \hline

\end{longtable}
\end{center}




*Через черту указана максимальная грузоподъемность полуприцепа, который может перевозить тягач.
**