\section{Медикаменты и яды}
\paragraph{Интоксикация.} Ядовитые повреждения считаются отдельно от остальных повреждений и отражают то, насколько много яда попало в организм и сколько еще он может сопротивляться яду.
\newline
Значение Интоксикации героя может быть больше, чем его максимальные ЕЗ.
\newline
Если значение Интоксикации превышает текущие ЕЗ героя, он становится Отравлен и на него накладываются эффекты Интоксикации \textit{всех} ядов, которые наносили ему Ядовитые повреждения. Для того, чтобы снять состояние Отравления, герой должен восстановить ЕЗ до уровня, когда значение Интоксикации становится меньше, чем текущие ЕЗ героя.
\newline
Во время отдыха герой понижает Интоксикацию на столько, сколько восстанавливает ЕЗ. Но если он применяет стимулятор или зелье, то он не снимает Интоксикацию, если это не указано в их описании. Так же если герой обращается за помощью к целителю, то Интоксикация снимаются отдельно.
\newline
Например, если у героя Иноксикация равна 5 и он получил 6 повреждений из других источников, Доку нужно потратить 11 зарядов своего Атрибута, чтобы полностью исцелить героя.
\paragraph{Первичный эффект:} Разовый эффект Лекарства или Яда. Для сопротивления эффекту герой должен преуспеть в проверки Вн против \textbf{|10+Токсичность|}.
\paragraph{Эффект Интоксикации:} целебные зелья, таинственные эликсиры и мощные стимуляторы зачастую являются не очень-то и полезными при частом применении. Каждое применение препаратов наносит при потреблении Ядовитые повреждения. Побочный эффект прекращается, как только Интоксикация становится меньше, чем его текущие ЕЗ.
\paragraph{Токсичность:} количество Ядовитых повреждений, которое наносит препарат при применении.
\paragraph{Отравленное оружие:} Если в названии яда есть свойство [масло], его можно нанести на оружие.
\newline При нанесении масла на оружие, он действует в течение \textbf{Тк} атак. Это правило действует как на оружие ближнего боя, так и на дальнобойное оружие. Для дальнобойного оружия это значит, что порция масла была распределена равномерно между боеприпасами. Если атака нанесла хотя бы 1 Пв, цель дополнительно получает Ядовитые повреждения в размере Токсичности Яда и должен сопротивляться его Первичному эффекту.
\paragraph{КУ Ядовитых повреждений.} Если атака, наносящая Ядовитые повреждения наносит критический удар, то тот яд, которым наносится атака сразу вызывает Эффект Интоксикации и не прекратит свое действие, пока с героя не будут сняты \textit{все} Ядовитые повреждения.

\subsection{Медикаменты}
\genAndGet{drugs}{meds}
\subsection{Яды}
\tbd