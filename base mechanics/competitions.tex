\section{СОСТЯЗАНИЯ}
Обычно для определения успеха или неудачи действия героя или статиста достаточно бросить кубик — все положительные и отрицательные факторы включены в бросок. Однако иногда мастер может добавить в ситуацию остроты. В этом случае и мастер, и игрок бросают кубики, прибавляют к ним все необходимые бонусы/штрафы и сравнивают результаты. В состязании побеждает тот, чей финальный результат окажется больше. При использовании этого правила замените в формулах, используемых для противостояния герою, 10 на бросок К20.
\newline
Если в Состязании противники получают равные результаты, то ни одна из сторон не может взять верх, и ситуация остается такой же, как и до броска. Рекомендуется использовать Состязание, только если герою противостоит персона или другой герой.
\newline
Если для состязания используется правило Быстрых проверок, то рекомендуется применять ее к стороне, против которой иницировали Состязание. Например, герой вызвал статиста на арм-реслинг. В этом случае герой совершает обычную Проверку Силы или Атлетики(СЛ), а статист - Быструю проверку Силы или Атлетики(СЛ).