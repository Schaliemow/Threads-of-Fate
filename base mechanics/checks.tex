\section{ПРОВЕРКИ}
\paragraph{Проверки} — это броски кубика К20, изображающие усилия героя, физические, волевые или умственные. Чем большее число выпало на кубике, тем больше шанс, что герой преуспеет. Например, когда герой совершает проверку Характеристики, игрок бросает К20 и прибавляет к выпавшему результату модификатор Характеристики. Различные факторы могут повысить или понизить шансы героя на успех. Они называются \textbf{бонусами} и \textbf{штрафами} и отображаются числами, которые прибавляются к результату броска или отнимаются от него.
\newline Проверки совершаются против фиксированного числа — \textbf{сложности}, которую задают мастер, контекст ситуации или правила. Если результат равен сложности или превышает ее, герой достиг успеха.
\newline Перед совершением проверки определите следующее:
\begin{itemize}
\item[--]Цель, которую герой пытается достичь совершением проверки.
\item[--]Сложность проверки, исходя из цели героя и контекста
ситуации, если она не определена правилами.
\item[--]Цену провала проверки, если она не определена правилами.
\item[--]Наличие Преимуществ и Помех.
\item[--]Наличие бонусов или штрафов.
\item[--]Допустимость Успеха с Неприятностями и потенциальные Неприятности, если он возможен.
\end{itemize}
\paragraph{Градации успеха:} некоторые проверки имеют градации успеха — например, проверки Доблести и Меткости. В этом случае важна величина разницы между результатом проверки и заданной сложностью.
\paragraph{Преимущества и Помехи:} порой обстоятельства складываются неблагоприятно или, наоборот, благоволят герою. В этом случае бросьте дополнительный кубик за каждую Помеху или Преимущество. Выберите меньший результат, если герой находится под действием Помехи, и больший, если герой обладает Преимуществом. Одновременное действие 1 Помехи и 1 Преимущества сводит их на нет. Герой не может страдать больше чем от 2 Помех за бросок, как не может реализовать больше 2 Преимуществ за бросок. То есть максимальное число кубиков в броске — 3.
\paragraph{Активные проверки} подразумевают некие действия героя — атаку, бег, прыжки, разговор, поиск, размышления. К активным не относятся проверки Наблюдательности, не связанные с целенаправленным поиском, проверки Воли, проверки на Потерю сознания, сопротивление яду и тому подобные.
{\paragraph{Коллективные проверки.} Когда нужно совершить одинаковые Проверки для большого количества существ, можно совершить Проверку один раз и добавить для каждого существа все бонусы и вычесть штрафы, чтобы определить результат.
\paragraph{Быстрые проверки.} В некоторых случаях, например, когда нужно отразить длительные усилия героя, повторяющиеся достаточно регулярно или для упрощения состязаний героев. Не кидайте кубик, вместо этого прибавьте 10 к значению Навыка. Дополнительно, прибавьте к значению 5, за каждое Преимущество, которым обладает герой, или отнимите 5, за каждую Помеху, от которой герой страдает. Вы можете применять статичные значения, когда герой стоит на часах, методично обшаривает стены в поисках потайного лаза или выполняет другие задачи, требующие систематического повторения одних и тех же действий.
\newline Например, если герой-часовой с Наблюдательностью +4 использует статичное значение этого Навыка, результатом его «проверки» всегда будет 14. Таким образом, во время дежурства он заметит все, для чего достаточно этого результата, но остальное ускользнет от его внимания. Это не обязательно подразумевает, что герой сконцентрировался на одном деле, — он может озираться по сторонам в поисках затаившихся врагов и при этом орудовать мечом. Разумеется, вы всегда можете использовать обычные проверки, если вам больше по нраву сюрпризы!
\paragraph{Критический провал и успех:} выпав на кубике, числа 1 и 20 отражают ошеломительные провалы и успехи на грани возможного. В таких случаях мастер может ввести в игру дополнительные эффекты броска, кроме неудачи или успеха. При выпадении 1 мастер может засчитать автоматический провал проверки, даже если бросок превысил сложность задачи. При выпадении 1 во время проверок Доблести или Меткости цель никогда не теряет Единицы Здоровья, даже если Доблесть или Меткость героя достаточно велики, чтобы поразить Защиту цели. При выпадении 20 на кубике мастер может засчитать автоматический успех проверки, даже если бросок не превысил сложность задачи. При выпадении 20 во время проверок Доблести или Меткости цель всегда теряет минимум 1 Единицу Здоровья, даже если Доблесть или Меткость героя недостаточно велики, чтобы поразить Защиту цели.
\newline Критические провалы профессионалов и Критические успехи дилетантов кому-то могут показаться нелогичными. С другой стороны, это мощный повествовательный инструмент, который не стоит игнорировать. Объяснив, из-за чего спасовал профи и преуспел дилетант, вы насытите вашу историю интереснейшими подробностями.