\section{ДОСУГ И РАЗВЛЕЧЕНИЯ}
Чем занимаются герои, когда выдается свободная минутка? Ответ на этот, казалось бы, незначительный вопрос может во многом определить развитие истории и наполнить ее событиями! Разумеется, вам не нужно применять эти таблицы, если ответ очевиден для всех участников игры. Эти правила призваны наполнить историю событиями как случайными, так и логически вытекающими из сюжетной канвы. Игроки и мастер могут использовать их, чтобы дать героям игромеханические преимущества или узнать что-то важное, а также черпать в них идеи для дальнейшего развития сюжета. Если у игрока нет конкретных идей, он может решить, как герой проведет свободное время, выбрав любой вариант из таблицы «Досуг и Развлечения».
\paragraph{Эффекты:} эффекты, описанные в таблице, входят в игру в следующей сцене и длятся до ее окончания, но мастер может сохранить их и на большее время, если это соответствует контексту. Разумеется, все материальные ценности, которые приобрел герой, останутся при нем. Суммарное число положительных эффектов Досуга и Развлечений не может превышать \textbf{|1 + МОб героя|}(минимум 1).
\paragraph{Проблемы:} возможные негативные последствия Досуга и Развлечений.
\paragraph{Риск:} вероятность того, что Досуг или Развлечение по самым разным причинам обернутся тоской и унынием. Чем выше Риск, тем больше шанс почувствовать в конце Досуга или Развлечения лишь усталость, пустоту и бессилие, даже если герой не пострадал физически и формально получил больше, чем потратил.
\paragraph{Сложность приобретения:} за развлечения приходится платить, да и самые обычные с виду занятия могут потребовать некоторых трат. Если герой совмещает несколько видов Досуга и Развлечений, сложите СП.
\paragraph{Скрытая угроза:} иногда самые невинные забавы могут закончиться сущим кошмаром! Некоторые виды Досуга и Развлечений предполагают проверки Скрытой угрозы. При этих проверках отнимите Риск Досуга и Развлечений от выпавшего значения для определения результата.
\paragraph{Всего да побольше:} виды Досуга и Развлечений могут совмещаться друг с другом. Проконсультируйтесь с мастером, чтобы выяснить, какие именно, хотя обычно это следует из логики ситуации. В любом случае герой может одновременно совмещать не более \textbf{|1 + ММд|} (минимум 2) видов Досуга и Развлечений. В случае необходимости используйте наибольший показатель Риска. Когда совмещенные виды Досуга и Развлечений следуют друг за другом, герой может потратить приобретенные эффекты на необходимые проверки Досуга и Развлечений. Если он не сделает этого, эффекты не считаются потерянными.
\paragraph{Затраченное время:} подразумевается, что герой уделяет Досугу или Развлечению не меньше 1 часа. Хотя, как правило, больше.
\paragraph{Сложность проверок:} сложность всех необходимых проверок равна \textbf{|10 + Риск|}. В некоторых случаях указанные проверки могут быть усложнены или заменены другими в соответствии с логикой ситуации.
\paragraph{Доступность:} совершите проверку Неприятностей, чтобы определить, доступен ли желаемый вид Развлечения. Разумеется, бросок совершается только в том случае, если у мастера и игроков есть какие-то сомнения на этот счет!
\trouble
{Массовая культура}%no sweat name
{Развлечение доступно и дешево. Используйте СП в таблице}%no sweat description
{Утеха для ценителей}%tough day name
{Развлечение широко распространено, но не так уж доступно. Используйте увоенную СП в таблице.}%tough day description
{VIP залы}%we have trouble name
{Развлечение доступно, но в силу неких причин довольно дорого. Используйте утроенную СП в таблице.}%we have trouble description
{Частные клубы}%fiasco name
{Развлечение недоступно, хотя некоторые Атрибуты могут помочь герою отыскать желаемое. Используйте утроенную СП в таблице.}%fiasco description
\subsection{Cтруктура сцены досуга}
\begin{enumerate}
\item Определите доступность Досуга или Развлечения, а также сколько видов Досуга и Развлечений одновременно совмещает герой.
\item Совершите проверки, связанные с эффектами.
\item Совершите проверки, связанные с проблемами, если требуется.
\item Совершите проверку Скрытой угрозы, если требуется.
\end{enumerate}
\section{ВАРИАНТЫ ДОСУГА}
\genAndGet{leisure}
