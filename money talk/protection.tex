\section{Защита}
\paragraph{Бонус к Защите (БЗщ):} доспехи и щиты дают Бонус к Защите. Герой не может надеть 2 комплекта доспехов одновременно, но может одновременно использовать 2 щита. В этом случае он получает БЗщ обоих щитов.
\paragraph{Класс Защиты(КЗ):} Добавляет свое значение к Размеру существа для определения условий Подавляющего превосходства, не меняя его размер для определения других условий.
\paragraph{Требуемая Выносливость(тВн):} Герой не может носить доспех или щит, если не обладает достаточной для этого Выносливостью. Сила не играет здесь важной роли — герой может поднять доспех и выдержать его вес, но выдохнется после нескольких минут активных действий. Разумеется, он может облачиться в тяжелый доспех, чтобы повыпендриваться перед девчонками или попозировать фотографу, но в бою представляет собой жалкое зрелище.
\newline
Если Выносливость героя меньше, чем тВн защиты, все физические проверки героя совершаются с Помехой, а все атаки по нему совершаются с Преимуществом.
\paragraph{ограничение модификатора Ловкости(оМЛв):} доспехи и щиты — тяжелые громоздкие конструкции, которые сильно ограничивают и сковывают движения. Некоторые доспехи и щиты ограничивают модификатор Ловкости, доступный герою для Защиты и проверок. Так, герой с 20 Лв (модификатор Лв +5), надевший кольчугу, будет добавлять к Защите и проверкам Ловкости и Навыков, основанных на Ловкости, лишь +3, потому что ограничение МЛв кольчуги составляет +3. Обратите внимание, что атаки любых видов основаны на Ловкости, так как модификатор Лв входит в состав Доблести и Меткости.
\paragraph{Количество помех для проверок Скрытности(ПС):} защита может шуметь при движении - звон сочленений, трение защитных пластин друг о друга, гул силовой установки в энергетических доспехах, это все мешает герою эффективно скрывать свое присутствие.
\paragraph{Надевание и снятие доспехов и щитов:} время, необходимое для этого, зависит от максимального модификатора Ловкости доспеха или щита. Если доспех или щит позволяет Любой модификатор Ловкости, время одевания и снятия составляет 1 минуту. Если максимальный модификатор Ловкости доспеха +2 или выше, время одевания составляет 5 минут, а время снятия составляет 1 минуту.
\newline
Если максимальный модификатор Ловкости +1 или ниже, время одевания составляет 10 минут (оно может быть сокращено до 5 минут, если кто-то помогает герою надевать доспех), а время снятия составляет 5 минут.
\paragraph{Шипы} на доспехе или щите позволяют эффективнее атаковать противника и охладить пыл любителей пообниматься. Шипы увеличивают параметр Выносливости, необходимой для ношения, на 1. Герой в шипованном доспехе, впечатанный в стену, может застрять в ней!
\paragraph{Единицы Здоровья} защиты равны \textbf{Вн}, необходимой для его ношения.
\paragraph{Прочность} защиты равна половине ее \textbf{ЕЗ}. Прочность вычитается из Единиц Здоровья, которые теряет защита.
\paragraph{Сопротивление к Повреждениям:} доспехи и щиты обладают Иммунитетом к Ядовитым Пв, Сопротивлением к Колющим, Проникающим, Огненным и Ледяным Пв и Уязвимы к Едким Пв.
\paragraph{Атаки по доспехам и щитам:} доспех или щит могут быть выбраны, как зона поражения (подробнее смотрите маневр «Сломать снаряжение» в разделе «Маневры»). Носитель получает Пв только в том случае, если доспех или щит получают Пв, которые не смогли поглотить их Прч и ЕЗ! Если доспех или щит уничтожены, носитель теряет их БЗщ, однако его МЛв все еще ограничен максМЛв доспеха или щита.
\paragraph{Удар щитом:} минимальная сила для ударов щитом равна минимальной Выносливости для ношения щита. Возможна ситуация, в которой герой может носить щит, но не может им бить.
\paragraph{Ширпотреб, индивидуальный заказ и шедевр:} дешевый, кое-как изготовленный доспех (вроде тех, в которых щеголяет уличная шпана) тяжел и неудобен. Выносливость, необходимая для его ношения, увеличивается на 1, а максимальный модификатор Ловкости понижается на 1. Если доспех не ограничивал модификатор Ловкости, то он получает максимальный модификатор в +4 и Осечку 5.
\newline
Доспех, изготовленный по индивидуальному заказы, безупречно подходит заказчику. Выносливость, необходимая для его ношения, уменьшается на 1, а максимальный модификатор Ловкости повышается на 1. Щиты в дополнение к этому повышают ЕЗ × 2 и Прочность × 2. Повысьте СП на 5.
\newline
Шедевр представляет собой безупречный экземпляр мастерской работы и получает все преимущества работы мастера. В дополнение, бонусы к Защите доспеха или щита возрастает на 1. Повысьте СП на 10.
\newline
Индивидуальный заказ и шедевр — штучные изделия. Все, кроме хозяина, при ношении получают эффекты ширпотреба, пока доспех не будет подогнан сведущим технарем под нового  ладельца.
\paragraph{Символ «*»} обозначает качества доспехов и щитов, не указанные в таблице и перечисленные в описательном блоке.
\paragraph{Символ «Ф»} обозначает доспехи, уместные лишь в фантастическом антураже.
\subsection{Броня}
\genAndGet{armor}

\subsection{Щиты}
\genAndGet{shields}


\subsection{Защита для больших и маленьких существ}
Повысьте СП доспеха или щита на 2 за каждую категорию размера больше Среднего. Стоимость доспехов и щитов для маленьких существ не меняется — меньшая стоимость материалов компенсируется сложностью работы. Увеличение СП за размер складывается с изменением СП за Халтуру, Работу мастера или Шедевр.
\paragraph{ЕЗ доспехов и щитов для больших и маленьких существ:} повысьте ЕЗ доспеха или щита в 1.5 раза за каждую категорию размера больше Среднего и понизьте в 1.5 раза за каждую категорию размера меньше Среднего. Например, латы для Среднего человека имеют 15 ЕЗ и Прч 7. Латы для Маленького полурослика будут иметь 10 ЕЗ и Прч 5, а латы для Большого огра — 22 ЕЗ и Прч 11!