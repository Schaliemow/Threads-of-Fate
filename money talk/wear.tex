\subsection{Износ и ремонт снаряжения}
Даже самое надежное устройство при ненадлежащей эксплуатации и недостаточном уходе начнет сбоить.
\paragraph{Проверка Износа} является Проверкой неприятностей, которая определяет, насколько ухудшилось состояние снаряжения во время эксплуатации.
\begin{tcolorbox}
Обычно, проверки Износа явно указаны в описании ситуаций, требующих их. Однако ведущий может потребовать проверку Износа в конце Сцены, если по его мнению герой слишком уж сильно издевается над своей экипировкой.
\end{tcolorbox}
\trouble
{Надежная штука}%no sweat name
{Устройство счастливо избежало неполадок и может быть использовано в дальнейшем.}%no sweat description
{Заело}%tough day name
{Небольшая проблема в механизме. Любой герой разберется с ней за 10 минут. Для того чтобы вернуть работоспособность устройства, совершите проверку Эксплуатации против 15. Устройство может быть использовано после проверки в любом случае, но в случае провала проверки его Осечка возрастает на 1.}%tough day description
{Заклинило}%we have trouble name
{Серьезная проблема в механизме. Для того чтобы вернуть работоспособность устройства, совершите проверку Эксплуатации против 20. Устройство может быть использовано после проверки в любом случае, но в случае провала проверки его Осечка возрастает на 2, а в случае успеха - на 1.}%we have trouble description
{Капитальная поломка.}%fiasco name
{Устройство не может быть использовано до ремонта, а его Осечка возрастает на 5.}%fiasco description
\paragraph{Ремонт Осечек} снаряжения доступен только в том случае, если Осечка была получена из-за Износа. Если Осечка является свойством нового снаряжения, избавиться от нее может только Изобретатель с помощью хода Эврика.
\newline
При ремонте осечек снаряжения герой должен совершить проверку Ремонта против \textbf{|15+Осечка|} и потратить материалы с СП, равныой величине Осечки. В случае провала проверки материалы не возвращаются. В случае критического провала снаряжение приходит в негодность и больше не полежит ремонту.
\begin{tcolorbox}
Если герой ремонтирует снаряжение, которое имело свойство Осечки до того, как она возрасла во время эксплуатации, герою все равно нужно учитывать в формулах полную Осечку снаряжения. Это значит, что механизм устройства настолько сложный, что его непросто ремонтировать.
\end{tcolorbox}
\paragraph{Ремонт Потери ЕЗ} снаряжения требует совершения проверки Ремонта против \textbf{|15+Потерянные ЕЗ|} и потратить материалы с СП, равныой величине количеству потерянных ЕЗ. В случае провала проверки материалы не возвращаются. В случае критического провала снаряжение приходит в негодность и больше не полежит ремонту.
\paragraph{Время изготовления:} за каждую еденицу сложности проверки выше 15 герой или статист должен потратить 1 сутки, занимаясь ремонтом желаемого предмета. Он может отвлекаться на другие дела и приключения, но должен провести не менее 4 часов за ремонтом. Ускорение ремонта невозможно, потому что это дело не терпит спешки.
