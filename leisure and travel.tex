\chapter{ДОСУГ И ПУТЕШЕСТВИЯ}
Здесь вы узнаете, как герои отдыхают, чем занимаются в свободное от приключений время, и что за опасности подстерегают их в путешествиях.
\section{СКРЫТАЯ УГРОЗА}
Опасности могут поджидать героев на каждом шагу. Пробираясь сквозь дремучую чащу, прогуливаясь по незнакомому городу, сделав привал на лесной поляне или же развлекаясь в большом казино, герои могут наткнуться на серьезные проблемы, если не проявят бдительность. В отличае от обычной проверки Неприятностей, проверка Скрытой угрозы обычно производится в начале Сцены и не приводит к немедленным и очевидным проблемам для героев. Эффект Скрытой угрозы может быть отложен на несколько сцен, а если ведущий не забудет, то и несколько сессий.
\trouble
{Неожиданная слава}%no sweat name
{Герои будут вознаграждены за смелость, отзывчивость и доброту, а нерешительность, равнодушие и жестокость не возымеют далеко идущих последствий.}%no sweat description
{Круги на воде}%tough day name
{Действия героев не приведут к значительным последствиям. Полученные знакомства мимолетны, враги незлопамятны, а хозяева вещей, которые герои прибрали к рукам нескоро заметят пропажу.}%tough day description
{Тень затмения}%we have trouble name
{Проблема, которую все же реально заметить, пока не станет слишком поздно. Сцена еще может обернуться сущим кошмаром, но герои выйдут сухими из воды, если не будут хлопать ушами.}%we have trouble description
{Петля на шее}%fiasco name
{Герои попали в переплет. Убитые разбойники имели влиятельных покровителей, найденные предметы были кем-то спрятаны, а спасенная красотка обокрала караван и сбежала!}%fiasco description
\section{ВСТРЕЧИ И НАХОДКИ}
Никто заранее не знает, когда герои на своем пути встретят неожиданную компанию или занимательную ситуацию. Даже просто гуляя по крупному городу они могут попасть в гущу событий, не говоря уж об исследовании таинственных руин!
\newline Проверка Встреч и Находок является проверкой Неприятностей, в которой дополнительно учитывается четность выпавшего числа. В таблице указаны возможные наполнения сцены в соответствии с проверкой.
\begin{tcolorbox}
Эта проверка требует броска кубика, даже если игроки принимают Каприз Судьбы. Проверка в этом случае определяет, какое из двух зол встречают герои: конфликтную Встречу или опасную Находку.
\newline При использовании Хода \textit{Повезло!} игроки могут сами решить, какая сцена их ждет: приятная Встреча или полезная Находка.
\end{tcolorbox}
\begin{center}
\begin{tabular}{|p{3cm}|p{6.5cm}|p{6.5cm}|}
\hline
\textbf{Результат проверки Неприятностей} & \textbf{Встречи(нечетный результат)} & \textbf{Находки(нечетный результат)}
\\ \hline
\textbf{Легкая добыча}\newline\textit{(19-20)} & Встреченные люди практически беззащитны. Отряд героев может помочь им или отобрать то, что они имеют. Все равно они не смогут оказать сопротивления. & Находка сулит богатства. Недавно покинутый дом или оставленный транспорт. Скорее всего там есть, чем поживиться и, похоже, рядом нет никого, кто мог бы помешать отряду героев.
\\ \hline
\textbf{Место интереса}\newline\textit{(13-18)} & Встреча нейтральна. Караван хорошо вооружен, но не собирается мешать отряду героев. Возможно, они захотят поболтать и обменяться товарами и информацией. & Занятное местечко. Давно оставленный лагерь, полуразрушенный дом или полусгнившая телега. Если внимательно осмотреть место, возможно можно найти полезную вещицу. Но останется вопрос - а оно того стоило?
\\ \hline
\textbf{Напряженная ситуация}\newline\textit{(7-12)} & Встреча на грани конфликта. Отряд героев встретил напуганных или разъяренных чем-то людей, а возможно прервали интимный момент раздела добычи над поверженным соперником. В любом случае, если герои не проявят тактичность, дальше будет говорить оружие. & Находка кричит об опасности. Разграбленный караван, место битвы или разодранный животными труп. Возможно, среди останков можно найти что-нибудь ценное, но нет никаких гарантий, что отряд героев не разделит участь павших.
\\ \hline
\textbf{Заваруха}\newline\textit{(1-6)} & К оружию! Нападение, засада, прямая конфронтация. Для того, чтобы мирно урегулировать эту Встречу придется расстаться с имуществом, обладать невероятным красноречием или надеяться на вмешательство Судьбы! & Находка приносит лишь беды. Опасный предмет, разрушенный мост или заваленный камнями перевал, кровавый алтарь со свежим жертвоприношением. Героям не получится пройти мимо, закрыв на происходящее глаза и они это знают
\\ \hline
\end{tabular}
\end{center}
\section{ОТДЫХ}
Восстановление Единиц Здоровья и Единиц Характеристик: во время 8-часового отдыха герой восстанавливает число ЕЗ и/или ЕХ, равное МВн (минимум 1). ЕЗ и ЕХ восстанавливаются в любых комбинациях в пределах восстанавливаемого значения. Если при этом герой получает квалифицированный медицинский уход, он восстанавливает в 2 раза больше ЕЗ и/или ЕХ (минимум 2). У лекаря должно быть хотя бы 1 Очко опыта во Врачевании, и он должен потратить на уход за раненым не меньше 4 часов. Один лекарь может одновременно присматривать за числом раненых, равным его МИн (минимум 1).
Сломанные конечности требуют особого внимания. Чтобы герой восстановил ЕЗ, потерянные при Переломе конечности, лекарь должен пройти проверку Врачевания против 15.
\paragraph{Отдых и доспехи:} герой может полноценно отдыхать (и восстанавливать ЕЗ и/или ЕХ), пока одет в доспех с БДЗщ +4 или меньше. Герой, по каким-то причинам отдыхавший в доспехе с большим БДЗщ, Устает. Обратите внимание, что ему все же удается худо-бедно поспать, а потому шанс задремать на посту или за столом в таверне существенно ниже, чем без сна вообще.
\section{ИЗГОТОВЛЕНИЕ ОРУЖИЯ, ДОСПЕХОВ И МЕХАНИЗМОВ}
Ремесленники редко путешествуют и участвуют в авантюрах. Им просто некогда этим заниматься — на изготовление добротного оружия или доспеха может уйти куча времени! Но если в вашей истории герои имеют достаточно свободного времени и оборудованную мастерскую, используйте таблицу ниже. Для изготовления снаряжения используется навык Наука.
\paragraph{Время изготовления:} за каждую еденицу сложности проверки герой или статист должен потратить 1 сутки, занимаясь изготовлением желаемого предмета. Он может отвлекаться на другие дела и приключения, но должен провести не менее 6 часов за делом. Ускорение изготовления невозможно, потому что это дело не терпит спешки.
\begin{center}
\begin{tabular}{|p{10cm}|c|}
\hline
Предмет & Сложность \\ \hline
Оружие/доспех/щит & 15 + БПв/БЗщ \\ \hline
Оружие наносит Колющие Повреждения & +5 \\ \hline
Доспех имеет БДЗщ 5 и больше & +5 \\ \hline
Доспех/щит с шипами & +5 \\ \hline
Примитивное устройство (мельница, влагоуловитель) & 10 \\ \hline
Простое механическое устройство (замок, музыкальная шкатулка) & 20 \\ \hline
Механическое устройство средней сложности (механическое опахало, башенные часы) & 25 \\ \hline
Сложное механическое устройство (двигатель внутреннего сгорания) & 30 \\ \hline
Конструкция комплексная(например, автомобиль включает в себя ходовую часть, корпус и так далее) & +10 \\ \hline
Медицинские препараты & (10+СП)/2 \\ \hline
Взрывчатка и Гранаты & (10+СП)/2 \\ \hline
Боеприпасы & СП/2 \\ \hline
Затраченное время увеличивается в 5 раз & -5 \\ \hline
Затраченное время увеличивается в 10 раз & -10 \\ \hline
Мастер не имеет доступа к оборудованной мастерской и работает подручными инструментами & Помеха на проверку \\ \hline
\end{tabular}
\end{center}

\section{ДОСУГ И РАЗВЛЕЧЕНИЯ}
Чем занимаются герои, когда выдается свободная минутка? Ответ на этот, казалось бы, незначительный вопрос может во многом определить развитие истории и наполнить ее событиями! Разумеется, вам не нужно применять эти таблицы, если ответ очевиден для всех участников игры. Эти правила призваны наполнить историю событиями как случайными, так и логически вытекающими из сюжетной канвы. Игроки и мастер могут использовать их, чтобы дать героям игромеханические преимущества или узнать что-то важное, а также черпать в них идеи для дальнейшего развития сюжета. Если у игрока нет конкретных идей, он может решить, как герой проведет свободное время, выбрав любой вариант из таблицы «Досуг и Развлечения».
\paragraph{Эффекты:} эффекты, описанные в таблице, входят в игру в следующей сцене и длятся до ее окончания, но мастер может сохранить их и на большее время, если это соответствует контексту. Разумеется, все материальные ценности, которые приобрел герой, останутся при нем. Суммарное число положительных эффектов Досуга и Развлечений не может превышать \textbf{|1 + МОб героя|}(минимум 1).
\paragraph{Проблемы:} возможные негативные последствия Досуга и Развлечений.
\paragraph{Риск:} вероятность того, что Досуг или Развлечение по самым разным причинам обернутся тоской и унынием. Чем выше Риск, тем больше шанс почувствовать в конце Досуга или Развлечения лишь усталость, пустоту и бессилие, даже если герой не пострадал физически и формально получил больше, чем потратил.
\paragraph{Сложность приобретения:} за развлечения приходится платить, да и самые обычные с виду занятия могут потребовать некоторых трат. Если герой совмещает несколько видов Досуга и Развлечений, сложите СП.
{\textcolor{orange}\paragraph{Скрытая угроза:} иногда самые невинные забавы могут закончиться сущим кошмаром! Некоторые виды Досуга и Развлечений предполагают проверки Скрытой угрозы. При этих проверках отнимите Риск Досуга и Развлечений от выпавшего значения для определения результата.}
\paragraph{Всего да побольше:} виды Досуга и Развлечений могут совмещаться друг с другом. Проконсультируйтесь с мастером, чтобы выяснить, какие именно, хотя обычно это следует из логики ситуации. В любом случае герой может одновременно совмещать не более \textbf{|1 + ММд|} (минимум 2) видов Досуга и Развлечений. В случае необходимости используйте наибольший показатель Риска. Когда совмещенные виды Досуга и Развлечений следуют друг за другом, герой может потратить приобретенные эффекты на необходимые проверки Досуга и Развлечений. Если он не сделает этого, эффекты не считаются потерянными.
\paragraph{Затраченное время:} подразумевается, что герой уделяет Досугу или Развлечению не меньше 1 часа. Хотя, как правило, больше.
\paragraph{Сложность проверок:} сложность всех необходимых проверок равна \textbf{|10 + Риск|}. В некоторых случаях указанные проверки могут быть усложнены или заменены другими в соответствии с логикой ситуации.
\paragraph{Доступность:} совершите проверку Неприятностей, чтобы определить, доступен ли желаемый вид Развлечения. Разумеется, бросок совершается только в том случае, если у мастера и игроков есть какие-то сомнения на этот счет!
\trouble
{Массовая культура}%no sweat name
{Развлечение доступно и дешево. Используйте СП в таблице}%no sweat description
{Утеха для ценителей}%tough day name
{Развлечение широко распространено, но не так уж доступно. Используйте увоенную СП в таблице.}%tough day description
{VIP залы}%we have trouble name
{Развлечение доступно, но в силу неких причин довольно дорого. Используйте утроенную СП в таблице.}%we have trouble description
{Частные клубы}%fiasco name
{Развлечение недоступно, хотя некоторые Атрибуты могут помочь герою отыскать желаемое. Используйте утроенную СП в таблице.}%fiasco description
\subsection{Cтруктура сцены досуга}
\begin{enumerate}
\item Определите доступность Досуга или Развлечения, а также сколько видов Досуга и Развлечений одновременно совмещает герой.
\item Совершите проверки, связанные с эффектами.
\item Совершите проверки, связанные с проблемами, если требуется.
\item Совершите проверку Скрытой угрозы, если требуется.
\end{enumerate}
\section{ВАРИАНТЫ ДОСУГА}
\genAndGet{leisure}

\section{ПУТЕШЕСТВИЯ}
Путешествия — один из основополагающих элементов многих жанров. Какая бы причина ни вынудила героев двинуться в путь, в дороге их подстерегает немало трудностей. Если путешествия являются важной частью вашей истории и вы желаете выяснить, насколько сложен будет путь, следуйте пунктам ниже:
\begin{enumerate}
\item \textbf{Сделал дело — гуляй смело.} Игроки должны решить, кто из героев или статистов в караване станет:
\begin{itemize}
\item \textbf{Навигатором}, который ищет безопасный путь, определяет подходящие места для стоянки и пополнения припасов. Навык «Выживание» — важнейший для Навигатора.
\item \textbf{Проводнком}, который проведет караван мимо нежелательных встреч и внезапных препятствий. Ему понадобятся Скрытность и Наблюдательность. Эта роль не применяется, если отряд передвигается Маршем. 
\item \textbf{Разведчиком}, который идет впереди каравана, отслеживая все подозрительное и разыскивая места, представляющие интерес. Ему понадобятся Наблюдательность.
\item \textbf{Механиком}, который следит за состоянием машин и прочей техники в группе. Ему потребуется Эксплуатация (Ин, Мд).
\item \textbf{Погонщиком}, который следит за тем, чтобы вьючные животные не провалились в яму и не наелись ядовитой травы, а техника. Он использует Обращение с животными.
\end{itemize}
\begin{tcolorbox}
Роли Механика и Погонщика являются опциональными. Если в отряде нет техники и въючных животных, Механик и Погонщик будут слоняться без дела и эти роли можно не назначать.
\end{tcolorbox}
Одну роль могут выполнять несколько героев (используйте правила Взаимопомощи), но один герой не может выполнять несколько ролей.
\newline
Когда по тем или иным причинам роль остается невыполненной, последствия могут быть самыми плачевными.
\paragraph{Если Навигатора, Погонщика(при условии, что в отряде есть вьючные животные) или Техника(при условии, что в отряде есть техника)} нет в караване, считайте, что при соответствующих проверках выпало 5. Если разница между этим результатом и целевой сложностью проверки превышает 10, считайте проверку Критическим провалом.
\paragraph{Если в караване нет Разведчика,} герои могут стать жертвами случайного нападения. Они уязвимы для всех тех опасностей, которые легко предотвратить, заметив вовремя! Сцены Встреч и Находок начинаются сразу после того, как определены их тип и Скрытая Угроза. В Боевых сценах отряд действует, как будто подвергся внезапному нападению.
\paragraph{Если в караване нет Проводника,} Отряд не может избежать Встреч и Находок и обязан принять участие в сценах, с ними связанных.
\item \textbf{По дороге всегда быстрее.} Определите \textbf{Опасность местности(ОМ)}, по которой предстоит пройти героям – в пути их ждет немало сюрпризов! Если в пути герои преодолевают местность с разными уровнями Опасности, используйте наибольший.
\begin{center}
\begin{tabular}{|p{10cm}|c|}
\hline
Тип местности & Опасность \\ \hline
Обжитые пригороды, фермерские угодья, торговые тракты, области, подробно и точно нанесенные на карты. & 0 \\ \hline
Прерии, равнины, области, не слишком подробно нанесенные на карты. & 1 \\ \hline
Лесистые и болотистые равнины, холмы. & 2 \\ \hline
Лесные дебри, топи, скалистые холмы, руины больших городов. & 3 \\ \hline
Горы и пустыни. & 4 \\ \hline
Джунгли и заболоченная чаща. & 5 \\ \hline
\end{tabular}
\end{center}
\item \textbf{Долго ли, коротко ли…} Определите длительность пути в днях, ориентируясь на скорость самого медленного транспорта каравана. Длительность задает базовую Сложность пути. Прибавьте к ней Опасность местности. Получившееся число — финальная Сложность пути.
\begin{center}
\begin{tabular}{|c|c|}
\hline
Длительность & Сложность пути \\ \hline
1-15 дней & 10 + ОМ \\ \hline
16-40 дней & 15 + ОМ \\ \hline
41 день и больше & 20 + ОМ \\ \hline
\end{tabular}
\end{center}
\begin{tcolorbox}
Для простоты определения Длительности путешествия считайте, что рельеф местности не влияет на скорость путешествия.
\end{tcolorbox}
\newline \textbf{Марш:} герои могут увеличить скорость передвижения вдвое. При этом они получают Помеху на проверки Наблюдательности и должны совершить проверку Вн или Атлетики (Вн) против 15. В случае провала герои измотаны и находятся в состоянии Усталости до тех пор, пока не отдохнут минимум 8 часов. Опасность местности на марше возрастает на 1. Если в караване есть транспортные средства, скакуны или вьючные животные, опасность местности возрастает на 2.
\newline \textbf{Тише едешь — дальше будешь:} герои могут ополовинить скорость передвижения и получить преимущество на проверки Скрытности Проводника для того, чтобы избежать любых Встреч и Находок.




\item \textbf{Да что вы, ребята, я сам здесь впервой!} Навигатор совершает проверку Выживания (Ин), Механик совершает проверку Эксплуатации (Ин, Мд), Погонщик совершает проверку Обращения с животными (Сл, Ин, Мд, Об) против финальной Сложности пути.
\newline
Успех проверок означает, что путешествие проходит без осложнений и позволяет героям наслаждаться относительным комфортом:
\begin{center}
\begin{tabular}{|c|p{10cm}|}
\hline
Величина успеха & Эффект \\ \hline
1-5 & - \\ \hline
6-10 & Расход еды, воды и других ресурсов сокращается вдвое. \\ \hline
11-15 & Караван прибывает на 1 день раньше. \\ \hline
16-20 & Караван может избежать 1 Встречи или Находки или совершить 1 дополнительный бросок по таблице Встреч и Находок. Караван прибывает на 2 дня раньше. \\ \hline
21 и больше & Караван может избежать до 2 Встреч или Находок или совершить до 2 дополнительных бросков по таблице Встреч и Находок. Караван прибывает на 3 дня раньше. \\ \hline
Критический Успех & Преимущество на проверку Встреч и Находок. \\ \hline
\end{tabular}
\end{center}
Если проверка провалена, то путешественники получают Повреждения, а Опасность Встреч и Находок, изначально равная \textbf{нулю}, возрастает. При провале или при успехе нескольких проверок величина провала/успеха суммируется. Например, если Навигатор провалил проверку на 5, а Погонщик – на 8, суммарная величина провала составит 13. Если Навигатор преуспел на 7, а погонщик провалил проверку на 4, суммарная величина успеха составит 3.
\newline \textbf{Цена провала.} При провале проверки, герои и транспорт получают повреждения, а Опасность Встреч и Находок возрастает. Сверьтесь с таблицей для того, чтобы определить последствия.
\newline \textbf{Загрязнение.} В некоторых ситуациях герои идут по местности, знаменитой своими токсичными испарениями, искажающими эманациями или дурманящей флорой. В этом случае потерянные ЕЗ при провале заменяются Интоксикацией.
\begin{center}
\begin{tabular}{|c|p{5cm}|p{5cm}|}
\hline
Величина провала & Потерянные героями ЕЗ & Опасность Встреч и Находок \\ \hline
1-5 & ОМ(мин 1) & 1 \\ \hline
6-10 & ОМ+2 & 2 \\ \hline
11-15 & ОМ+4 & 3 \\ \hline
16-20 & ОМ+7 & 4 \\ \hline
21 и больше & ОМ+10 & 5 \\ \hline
Критический Провал & Герои получают Интоксикацию даже в безопасных районах* & Помеха на проверки Скрытой угрозы \\ \hline
\end{tabular}
\end{center}
*в этом случае герои могли наестся ядовитых ягод, несъедобных грибов, выпить из зараженного источника или же устроить лагерь на поляне с дурманящими растениями.
\item \textbf{Встречи и Находки}. Приятные неожиданности редки в пути, зато других хоть отбавляй. Совершите проверку Неприятностей и определите число Встреч и Находок:
\begin{center}
\begin{tabular}{ |p{2.7cm}|p{12cm}| }
\hline
\textbf{Результат проверки Неприятностей} & \textbf{Количество Встреч и Находок}
\\ \hline
19-20 & \textbf{0}+Опасность Встреч и Находок
\\ \hline
13-18 & \textbf{1}+Опасность Встреч и Находок
\\ \hline
7-12 & \textbf{2}+Опасность Встреч и Находок
\\ \hline
1-6 & \textbf{3}+Опасность Встреч и Находок
\\ \hline
\end{tabular}
\end{center}
\textbf{Сложность Встречи.} Чтобы определить целевую сложность проверок, не связанных с Общением, сложите \textbf{|10 + ОМ + Опасность Встреч и Находок|}.

\item \textbf{Встречи и Находки} Совершив проверку Встреч и Находок, определите наполнение сцены. Обратите внимание, что проверка «Скрытой угрозы» все еще может серьезно изменить смысловое наполнение сцены.
\item \textbf{Скрытая угроза.} В начале сцены Встречи или Находки, совершите проверку Скрытой угрозы. Отнимите от результата проверки Опасность Встреч и Находок, определенную на 4 этапе. Скрытая угроза не обязательно проявится в начале сцены, но дает мастеру хорошее представление о том, чем она может закончится.
\item \textbf{А что это унас тут?} Для того, чтобы заранее заметить Встречу и не проморгать Находку, Разведчик отряда должен преуспеть в проверки Внимательности против Сложности Встречи.
\item \textbf{Я тут мимо проходил.} Если отряд желает избежать Встречи, Проводник отряда должен преуспеть в проверки Скрытности(Ин) против Сложности Встречи, чтобы провести союзников мимо, не привлекая внимания. В случае провала, сцена Встречи начинается и проверки Впечатления статистов на отряд совершаются с Помехой.
\newline
Если Отряд желает обойти Находку стороной, Проводник отряда должен преуспеть в проверки Наблюдательности(Мд) против Сложности Встречи, чтобы найти обходной путь и не попасть в возможную засаду.
\item \textbf{Поболтаем?} Если отряд не избежал встречи и проверки Впечатлений достаточно хороши и нет спешки, большинство статистов готовы общаться и торговать с героями. Проверки Впечатлений не принесут результатов лучше Доброжелательности, но действия героев — могут!
\end{enumerate}


\section{ПЕРЕНОС ТЯЖЕСТЕЙ}
Веса в игре измеряются в килограммах. Комфортная нагрузка для героя равна значению его \textbf{|Сл × 3|}. Если герой несет больший вес, то его Ск падает вдвое. В дополнение к этому, если нагрузка героя превышает параметр его \textbf{|Сл × 5|}, все его активные проверки совершаются с Помехой, а все атаки по нему совершаются с Преимуществом. Максимальная нагрузка героя, с которой он может идти, равна его \textbf{|Сл × 10|}.
\newline
Герой может толкать, тянуть и отрывать от земли вес, вдвое превышающий его максимальную нагрузку. Если герой толкает или тянет вес, превышающий его максимальную нагрузку, его Ск падает до 1.
\paragraph{Большие} герои могут нести больший вес. Герой увеличивает вдвое все параметры, связанные с переносом тяжестей, за каждую категорию размера больше Среднего. Маленькие герои ополовинивают эти параметры за каждую категорию размера меньше Среднего.
\paragraph{Четвероногие существа}, такие, как кони, мулы и слоны, способны переносить больший вес. Увеличьте вдвое все параметры, связанные с переносом тяжестей после учета бонусов или штрафов за размер. Например, чтобы определить комфортную нагрузку для Большого коня, умножьте его Сл на 3, затем на 2 за размер и еще на 2 — за четвероногость. Маленькая лайка при этом будет способна нести такой же вес, как Средний человек с аналогичным параметром Силы. Гигантские пауки, улитки и змеи считаются четвероногими для определения нагрузки!
\section{ПЛАВАНИЕ}
Вплавь герой передвигается с половиной своей Ск. Если герой несет вес, превышающий значение его Силы, ему потребуются проверки Сл или Атлетики (Сл), чтобы держаться на воде. Если герой плывет с весом, превышающим значение его \textbf{|Сл × 3|}, он получает Помеху на эту проверку. Если герой плывет с весом, превышающим значение его \textbf{|Сл × 5|}, он получает 2 Помехи. Герой автоматически проваливает проверку, если несет максимальный вес. Проваливший проверку герой находится в состоянии Удушья, хотя провал не всегда означает, что герой тонет. Используйте стандартную таблицу сложностей для отображения быстрого течения или неблагоприятных погодных условий. Иногда герой может передвигаться вплавь с полной Ск или быстрее, например, если он плывет по течению, хотя в таких случаях ему точно потребуются проверки Атлетики (Сл).
\newline
Герой, вынужденный сражаться во время плавания, совершает все атаки с Помехой. Если герой провалил проверку Сл или Атлетики (Сл) во время плавания, он пропускает свою Очередь!
\section{ПАДЕНИЕ}
{\textcolor{orange} Падая или прыгая с большой высоты, герой может пострадать. Герой может спрыгнуть с высоты в 3 метра без всякого вреда для себя. За первый метр после 3 герой получает 2 Пв, за каждый следующий метр полученные героем повреждения увеличиваются вдвое, то есть, спрыгнув с высоты 7 метров, герой получит 8 Пв. Если при прыжке герой получает Пв, достаточные для нанесения Опасной раны, то он ломает конечность — случайно определите какую. Успешная проверка Атлетики (Лв) против \textbf{|7 + общая высота падения|} избавит героя от Перелома, но не избавит его от Пв.}
\newline
Если герой падает, то есть не смог сгруппироваться и подготовиться к прыжку, то он дополнительно получает 2 Пв за каждый метр, который пролетел, а также число Пв, равное сумме БЗщ щита и доспеха, которые на нем надеты. Возможно, потребуется проверка Неприятностей, чтобы определить, отделался герой синяками или получил серьезные травмы.
\section{ЛОВУШКИ}
Герой попадает под действие ловушки, если не заметил ее вовремя при помощи проверки Наблюдательности (Мд) или активировал случайно, неудачно применив Наблюдательность (Ин). Опасность ловушки (как абстрактную, так и соответствующий параметр) определяет мастер. Многообразие ловушек слишком велико, чтобы подробно перечислять их здесь. Тем не менее, их можно разделить на несколько основных типов:
\begin{itemize}
\item \textbf{Импровизированные ловушки:} ловушки из подручных материалов — бесхитростные, но все еще смертельно опасные. Если герой подвергается действию ловушки, то получает Пв, равные \textbf{|Величине проверки Выживания (Ин) установившего ловушку — БАЗщ — Наблюдательность (Мд) — БДЗщ*|}.
\item \textbf{Ловушки, наносящие фиксированные Повреждения:} если герой подвергается действию ловушки, то получает Повреждения, равные \textbf{|Опасность ловушки — БАЗщ — БДЗщ*|}. Например, ловушка, изрыгающая пламя, имеет Опасность 30 и заполняет огнем область 3 × 3 метра. Если герой в чешуйчатом доспехе подвергнется ее действию, то получит 30 — 10 — 4 = 16 Пв.
\item \textbf{Ловушки с фиксированной Доблестью или Меткостью:} как правило, это ловушки с подвижными частями — дротики, выстреливающие из стен, или стальные лезвия, вылетающие из потолка. Ловушка обладает параметром Дб или Мт и в случае активации совершает проверку против Зщ героя по обычным правилам.
\item \textbf{Ловушки с ядами:} если герой подвергается действию ловушки, то в дополнение к прочим эффектам на него сразу накладывается действие яда, как при КУ отравляющими повреждениями.
\item \textbf{Несмертельные ловушки:} ловушки, которые могут захватывать, погружать в сон, наносить Несмертельные Пв или еще как-то выводить героя из строя вместо того, чтобы убивать его.
\item \textbf{Смертельные ловушки} не наносят Повреждений — если герой подвергся их действию, его ЕЗ сразу падают до 0. Скорее всего, у спутников жертвы будет совсем немного времени, чтобы вытащить изувеченное тело или то, что от него осталось. Например, герой упавший в яму с концентрированной кислотой, понижает свои ЕЗ до 0 и совершает проверку Вн против 15. Если он преуспевает, у спутников будет Круг на то, чтобы попытаться вытащить героя и помочь ему. В противном случае герой растворяется в кислоте.
\end{itemize}
\paragraph{*Ловушки и Бонус доспеха:} разумеется, крепкий доспех поможет спастись во многих случаях… Хотя если герой попал в яму с зыбучим песком, эта груда железа станет серьезной проблемой! Так или иначе, лучшая защита от ловушки — высокая Наблюдательность (Мд).