\section{ОСНОВНЫЕ ХАРАКТЕРИСТИКИ}
\paragraph{Сила (Сл)} показывает, насколько герой развит физически. От этой Характеристики зависит, какой вес может нести герой, и ущерб, который он причиняет оружием, использующим мускульную силу. Также параметр Силы влияет на то, какое оружие герой сможет успешно применить в принципе.
\paragraph{Ловкость (Лв)} отвечает за быстроту и координацию движений. Эта Характеристика так же важна для воина, как и Сила. Еще она пригодится герою, который собирается стрелять, срезать кошельки и карабкаться по деревьям.
\paragraph{Выносливость (Вн)} пригодится любому герою. Высокая Выносливость означает, что герой крепок, редко болеет и легко оправляется от ран. Также от этой Характеристики зависит, в какой броне может эффективно действовать герой.
\paragraph{Интеллект (Ин)} помогает запоминать информацию и учиться на своих и чужих ошибках. Интеллект необходим любому герою, который желает иметь высокие параметры Навыков — именно он задает пределы их роста.
\paragraph{Мудрость (Мд)} включает в себя находчивость, наблюдательность, здравомыслие и глубинные инстинкты. Именно Мудрость поможет герою вовремя заметить опасность… или просто избежать ее.
\paragraph{Обаяние (Об)} позволит наладить контакт с окружающими и понравиться им, не особенно усердствуя. Этот параметр незаменим для того, кто предпочитает действовать исподволь и добиваться своего без применения насилия. Помимо того, Обаяние определяет, в каком ключе герой воспринимает окружающий мир. Высокое Обаяние — залог оптимизма! Обаяние никак не связано с внешней привлекательностью героя, хотя обаятельные герои часто кажутся окружающим симпатичными.
\paragraph{Модификаторы Характеристик (М)} в большинстве формул используется не полное значение Основных характеристик, а \textbf{Модификатор = | (Основная характеристика — 10) ÷ 2|}.
\begin{center}
\begin{tabular}{ |c|c| }
\hline
\textbf{Основная характеристика} & \textbf{Модификатор}
\\ \hline
1 & -5
\\ \hline
2-3 & -4
\\ \hline
4-5 & -3
\\ \hline
6-7 & -2
\\ \hline
8-9 & -1
\\ \hline
10-11 & 0
\\ \hline
12-13 & 1
\\ \hline
14-15 & 2
\\ \hline
16-17 & 3
\\ \hline
18-19 & 4
\\ \hline
20 & 5
\\ \hline
\end{tabular}
\end{center}
Проверки Основных характеристик совершаются следующим образом:
\begin{enumerate}
\item Бросьте К20 и прибавьте к нему модификатор Основной характеристики.
\item Сравните получившееся число со сложностью проверки. Герой преуспевает, если число равно сложности или превышает ее.
\end{enumerate}