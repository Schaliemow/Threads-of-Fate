\section{БОЕВЫЕ ХАРАКТЕРИСТИКИ}
\paragraph{Повреждения (Пв):} результатом успешных проверок Боевых характеристик (за исключением Защиты) являются Повреждения — потеря Единиц Здоровья от атак и вредоносных эффектов.
\paragraph{Бонус к Повреждениям (БПв)} обычно дается оружием и Могуществам. Чем он выше, тем больше шансов у героя нанести цели Повреждения. Бонус к Повреждениям является частью Боевых характеристик, Доблести, и Меткости и Меткости Магической стрелы, и может изменяться в зависимости от того, какое оружие или способность использует герой.
\paragraph{Доблесть (Дб) = |Владение оружием/Рукопашный бой + МСл + МЛв + БПв|.} Чем выше Доблесть героя, тем он опаснее в ближнем бою. На Доблесть оказывают влияние два разных Навыка — Владение оружием и Рукопашный бой, в зависимости от того, сражается герой вооруженным или безоружным, и какие боевые Маневры он использует.
\paragraph{Меткость (Мт) = |Стрельба + МЛв + БПв|.} Чем выше Меткость, тем опаснее герой в дистанционном бою. Луки и метательное оружие также используют модификатор Силы (МСл) при подсчете БПв.
\paragraph{Защита (Зщ) = |Базовая защита + МЛв + бонус доспеха + бонус щита|.} Чем выше Защита, тем сложнее поразить героя атаками.
\begin{center}
\begin{tabular}{ |c|c| }
\hline
\textbf{Размер существа} & \textbf{Базовая защита}
\\ \hline
Крошечный & 12
\\ \hline
Маленький & 11
\\ \hline
Средний & 10
\\ \hline
Большой & 9
\\ \hline
Огромный & 8
\\ \hline
Громадный & 7
\\ \hline
\end{tabular}
\end{center}
\paragraph{Прочность (Прч):} камень, сталь, лед и многие другие материалы имеют показатель Прочности. Прочность вычитается из Повреждений, нанесенных цели. Некоторые существа — например, стальные големы и ожившие деревья — также обладают Прочностью! Прочность не является частью Боевых характеристик, но тесно связана с ними.
\paragraph{}Проверки Боевых характеристик:
\begin{enumerate}
\item Бросьте К20 и прибавьте к нему значение Доблести, Меткости или Меткости Магической стрелы героя.
\item Сравните получившееся число с Защитой цели. Цель получает 1 Повреждение за каждую 1, на которую атакующий герой преодолел Защиту цели.
\end{enumerate}
Например, если герой с Доблестью 10 атаковал статиста с Защитой 18 и на К20 выпало 14, статист получит 10 +14 — 18 = 6 Повреждений.
\newline
Подробнее о проверках Боевых характеристик читайте в разделе «Маневры».
\paragraph{Проверки Защиты:} в сценах, не подразумевающих
детализированных боевых действий — например, когда герой
прорывается сквозь разъяренную толпу, передвигается под беглым обстрелом или бежит по коридору, наполненному ловушками, мастер может определить полученные героем Повреждения при помощи проверки Защиты.
\begin{enumerate}
\item Бросьте К20 и прибавьте к нему значение \textbf{|Зщ — 10|}.
\item Сравните получившееся число со сложностью проверки. Герой получает 1 Повреждение за каждую 1, на которую провалил проверку.
\end{enumerate}
Например, герой Среднего размера с 16 Защитой передвигается
под беглым огнем вражеских стрелков. Мастер устанавливает
20 сложность проверки. На К20 выпадает 12. 12 +16 — 10 = 18.
Герой получает 2 Повреждения, так как 20 (Установленная
сложность) — 18 (Результат проверки) = 2.