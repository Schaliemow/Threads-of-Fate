\section{РАЗВИТИЕ ГЕРОЯ}
После начала игры герой может потратить заработанные Очки опыта следующим образом:
\begin{itemize}
\item[--] Повысить значение любого Навыка на 1, потратив 1 Очко опыта.
\item[--] Повысить Богатство на 1, потратив 1 Очко опыта.
\item[--] Изучить новый язык или наречие, потратив 1 Очко опыта.
\item[--] Получить любой Элемент, потратив 2 Очка опыта.
\item[--] Изучить новое заклинание, потратив 2 Очка опыта.
\item[--] Изучить новую алхимическую формулу, потратив 2 Очков опыта.
\item[--] Научиться обращению с новым оружием, потратив 2 Очка опыта.
\item[--] Повысить на 1 максимальные Единицы Здоровья, потратив 2 Очка опыта.
\item[--] Повысить на 1 максимальные Единицы Маны, потратив 2 Очка опыта.
\item[--] Изучить Трюк, потратив 5 Очков опыта.
\item[--] Повысить на 1 любую Основную или Вторичную характеристику, потратив 5 Очков опыта.
\item[--] Получить любой Атрибут, потратив 10 Очков опыта.
Повышение Основных характеристик приводит к повышению
Вторичных. Например, если герой приобрел 1 единицу Выносливости, то его ЕЗ также вырастут.
\end{itemize}
\paragraph{}
Новые способности не могут взяться из ниоткуда. Чтобы отобразить развитие героя, есть два способа:
\begin{enumerate}
\item Вы можете исходить из уже существующих внутриигровых фактов. Например, прежде чем приобрести Атрибут «Аристократ», герой проявил себя перед государем и заслужил титул, получил его путем вероломных интриг… или же просто купил за внушительную сумму денег. Если герой увеличивает Богатство или Владение оружием, то перед этим он разумно распоряжался своими средствами и принимал участие в битвах.
\item Вы можете создать факт, договорившись с мастером и соигроками о логичном внутриигровом обосновании приобретения вашего героя. Например, в случае Атрибута «Аристократ» герой может оказаться потерянным наследником древнего рода, повышение Богатства произошло за счет выплаты банковских процентов или выигрыша на скачках, а Владение оружием улучшилось благодаря тренировкам, на которые герой тратил свободное время между игровыми встречами.
\end{enumerate}