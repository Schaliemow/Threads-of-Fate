\section{РАЗМЕРЫ СУЩЕСТВ}
\paragraph{}
Во время приключений герои могут встретить множество существ, размером превышающих человека или ощутимо уступающих ему: от сказочных драконов и фей, до гиганских человекоподобных роботов и мозговых червей. За эталон принят человек ростом от 150 до 210 см. Он имеет Среднюю категорию размера. Размер не влияет на Основные характеристики существа напрямую, однако большие существа — легкая мишень для атак любого рода, а маленькие существа вынуждены использовать маленькое оружие (зачастую наносящее меньше ущерба). Герой-человек может по договоренности с мастером начать игру Маленьким (карлик) или Большим (великан).
\newline
Существа занимают определенную область в зависимости от своих размеров. Это вовсе не означает, что существо занимает эту область целиком (хотя бывает и такое). Несомненно одно — в этой области существо может и будет мешать передвижению недругов. Обратите внимание, что высота в холке четвероногих существ зачастую меньше, чем рост существ соответствующего размера, указанный в таблице.
\paragraph{Бонус/штраф размера (БШР):} размер, отличный от Среднего, может быть выгоден в одних ситуациях и мешать в других. Небольшому существу легче прятаться и избегать от ударов (это учитывается в его Базовой защите), а крупному — хватать и удерживать противника.
\newline
Небольшие существа получают положительный БШР при проверках Атлетики (Лв), Ловкости рук, Скрытности и в других ситуациях, когда компактный размер, малый вес и тонкие пальчики помогают преуспеть. Крупные существа в этих случаях получают отрицательный БШР.
\newline
И наоборот — крупные существа получают положительный БШР при Захватах, проверках Атлетики (Сл), попытках сбить противника с ног и в других ситуациях, когда массивное сложение и длинные руки помогают одержать верх. Маленькие существа в этих случаях получают отрицательный БШР.
\paragraph{Модификатор размера:} наравне с Выносливостью определяет, насколько существо восприимчиво к физическому урону. Некоторые способности учитывают размеры существ. У этих способностей есть свойство Размер Имеет Значение(РИЗ). В этом случае Большое существо считается как 2 Средних, Огромное — как 3 Средних, а Громадное — как 4!
\begin{center}
\begin{tabular}{ |c|c|c|c|c| }
\hline
Размер & Модификатор Размера & БШР & Возможный рост & Занимаемая область
\\ \hline
Крошечный(К) & 1 & -2/+2 & 0.01-0.65 метра & 0.5 × 0.5 метра
\\ \hline
Маленький(М) & 2 & -1/+1 & 0.66-1.49 метра & 1 × 1 метра
\\ \hline
Средний(С) & 3 & 0 & 1.5-2.1 метра & 2 × 2 метра
\\ \hline
Большой(Б) & 4 & +1/-1 & 2.2-3 метра & 3 × 3 метра
\\ \hline
Огромный(О) & 5 & +2/-2 & 3.01-9.99 метра & 5 × 5 метров
\\ \hline
Громадный(Г) & 6 & +3/-3 & 10 метров и больше & 7 × 7 метров или даже больше
\\ \hline
\end{tabular}
\end{center}