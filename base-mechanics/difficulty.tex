\section{ПРИМЕРНАЯ СЛОЖНОСТЬ ЗАДАЧ}
\begin{center}
\begin{tabular}{ |c|c| }
\hline
 \textbf{Задача} & \textbf{Сложность} \\ 
\hline
 Примитивная & 5 \\
\hline
 Повседневная & 10 \\
\hline
 Придется попотеть & 15 \\
\hline
 Работа для эксперта & 20 \\
\hline
 Вызов для эксперта & 25 \\
\hline
 На грани возможного & 30 \\
\hline
\end{tabular}
\end{center}


\paragraph{Успех с Неприятностями:} если герой не прошел проверку, игрок может предложить ввести в игру Неприятность, позволяющую тому преуспеть или сопутствующую успеху. Например, вор открыл замок, но старый механизм пронзительно заскрипел и разбудил стражника. Или воин поразил противника, но дешевый клинок сломался при ударе. Считайте, что герой добился минимально необходимого успеха и автоматически получил вариант «Катастрофа» при проверке Неприятностей. Не протягивайте к герою Нить — его наградой за Неприятность будет успех проверки. Ниже вы найдете возможные примеры Неприятностей, осложняющих успех проверки.
\begin{itemize}
\item[--] \textbf{Возможность для недругов:} успех героя позволяет недругам приблизиться к своей цели или даже достичь ее. Эта Неприятность может оставаться за кадром до тех пор, пока герои не столкнутся с ее последствиями.
\item[--] \textbf{Временные затраты:} выполнение задачи требует больше времени, чем планировал герой.
\item[--] \textbf{Герой под ударом:} герой преуспел, но оказался в затруднительном положении. Его жизнь, здоровье или репутация под угрозой!
\item[--] \textbf{Невыгодная позиция:} успех вынуждает героя занять невыгодную позицию. Несколько последовательных выборов этого варианта могут привести героя на край обрыва, в глухой тупик или под обстрел артиллерийской батареи!
\item[--] \textbf{Оповещение недругов:} выполнение задачи привлекает к герою нежелательное внимание.
\item[--] \textbf{Ослабление эффекта:} герой выполнил задачу, но в самом скором времени статус-кво будет восстановлен.
\item[--] \textbf{Перерасход ресурсов:} герой преуспел, но потратил гораздо больше ресурсов, чем планировал.
\item[--] \textbf{Поломка снаряжения:} герой справился с задачей, но его снаряжение пришло в негодность.
\item[--] \textbf{Союзники под ударом:} успех героя приводит к тому, что его товарищи оказываются в затруднительном положении. Их жизнь, здоровье или репутация под угрозой!
\item[--] Ущерб: герой добился своего, но получил Опасную рану. Герой теряет число ЕЗ, достаточное для получения Опасной раны, вне зависимости от имеющихся у него защитных средств.
\end{itemize}
\paragraph{}Если герой не распределил Очки опыта в Навык, проверку которого он совершает, выберите 1 дополнительную Неприятность из списка.
Если герой достигает успеха только при выпадении 20 на кубике (или правила не позволяют ему совершить проверку в принципе), выберите 1 дополнительную Неприятность из списка.
\paragraph{Успех с Неприятностями и Экспертные навыки:} герой может применять Успех с Неприятностями при использовании Экспертных навыков, к которым не имеет доступа, хотя фактически бросок кубика не совершается.
Мастер может запретить успех с Неприятностями, если, по его мнению, герой в ходе успеха приобретет значительно больше, чем потеряет, или если при проверке на кубике выпала 1. Успех с Неприятностями может быть применен до броска кубика.
\paragraph{\textit{Успехи с Неприятностями позволят героям преодолеть полосу невезения, а мастеру и игрокам — наблюдать за развитием сюжета, а не за бесконечной чередой неудач. Не ограничивайтесь вариантами из списка, опирайтесь на жанр и настроение вашей игры!}}
\paragraph{Взаимопомощь:} герои могут помогать друг другу, если логика ситуации это допускает. Выберите героя, который будет совершать основную проверку и определите тех, кто ему помогает. Определив сложность проверки, отнимите от нее 5 — это сложность задачи для помощников. Совершите проверку профильной Характеристики или Навыка для каждого из помощников. Если помощник преуспел в проверке, герой, совершающий основную проверку, получает Преимущество. Если помощник потерпел неудачу, герой, совершающий основную проверку, получает Помеху. Не забывайте, что единовременно герой не может иметь более 2 Преимуществ или 2 Помех на бросок.