\section{НЕПРИЯТНОСТИ}

Зачастую проблемы создают герои, но иногда Неприятности сами находят их. Неприятности изображают неблагоприятные события, которые могут случиться, а могут и пройти стороной. В потенциально опасной сцене, такой как прогулка по огромной свалке, обыск древнего убежища или прыжок в море с обрыва, мастер может инициировать проверку Неприятностей, если контекст ситуации недостаточно ясно говорит о том, что герою ничто не угрожает.
\newline
Неприятности всегда должны оставлять герою хотя бы мизерный шанс спастись и не должны убивать его сразу.
\trouble
{Успех}%success name
{Герой вышел сухим из воды. Ну, на то он и герой.}%success description
{Затруднения}%difficulties name
{Герой вовремя заметил надвигающиеся трудности. Скорее всего, он сумеет их избежать. Скорее всего…}%difficulties description
{Проблемы}%troubles name
{Герой оказался в сложном, но не безвыходном положении.}%troubles description
{Катастрофа}%fiasco name
{Герой на волосок от смерти. Из грязного кабака вывалилась толпа головорезов, трухлявый пол просел под ногами, а в воде притаились острые камни. Герою будет непросто выкрутиться.}%fiasco description
Это правило может использоваться и несколько иначе — для определения того, оказался ли под рукой у героя необходимый предмет, есть ли поблизости разыскиваемое героем заведение и так далее, если контекст не дает исчерпывающего ответа на этот вопрос.
\trouble
{Да}%success name
{Герой получает желаемое или может получить желаемое, приложив незначительные усилия или потратив немного ресурсов.}%success description
{Да, но}%difficulties name
{Герой может получить желаемое, приложив усилия или потратив ресурсы.}%difficulties description
{Нет, но}%troubles name
{Герой может получить желаемое, только если приложит серьезные усилия или потратит значительные ресурсы.}%troubles description
{Нет}%fiasco name
{Герой не может получить желаемое. Ему придется искать другие пути.}%fiasco description
В этом случае проверка неприятностей будет выглядеть так: Например, скрываясь от преследования, герой забирается в незнакомый дом и пытается найти там оружие. Проверка Неприятностей покажет, держит ли хозяин в доме хоть что-то, похожее на оружие. В противном случае герою придется довольствоваться посудой или ломать мебель для того, чтобы вооружиться!
\paragraph{Общие Неприятности:} иногда возникают ситуации, в которых Неприятности напрямую касаются всех участников сцены — например, если в корабль дал течь в шторм, или шериф считает всех героев одинаково виновными. В этом случае при Капризе Судьбы «Катастрофа!», инициированном мастером, Нити протягиваются ко всем героям и персонам, присутствующим в сцене. Откупиться от Каприза можно по обычным правилам.
\paragraph{Неприятности под Контролем:} обычно Неприятности случаются внезапно, но иногда герою выпадает шанс смягчить (или усугубить) эффект. Если мастер считает, что герой может как-то повлиять на результат, герой может совершить проверку Характеристики или Навыка, уместного в контексте ситуации. Сложность проверки определяет мастер, но не рекомендуется устанавливать ее выше 20. Величина успеха прибавляется к результату проверки Неприятностей, а величина провала вычитается из него. Например, в трущобах герой может постараться не мозолить окружающим глаза и использовать Скрытность против сложности, установленной мастером. Это все еще не помешает герою случайно наткнуться на грабителей, но может \textit{понизить шанс} такой встречи. Если герой прошел Скрытность на 3, то к числу, выпавшему при проверке Неприятностей, будет прибавляться 3. Если герой провалил Скрытность на 2, из числа, выпавшего при проверке Неприятностей, будет вычтено 2, а его неумелые попытки выглядеть незаметно привлекут внимание окружающих.
\paragraph{Когда использовать проверку Неприятностей?} Эта механика позволяет мастеру создавать игровые события и факты без предварительной подготовки, руководствуясь контекстом сцены, и при этом разделять повествовательные права с игроками. Она не заменяет собой проверки Навыков (хотя ситуации, в которых такая замена будет уместна, могут возникнуть). Проверка Неприятностей позволит легко и быстро узнать, в каком настроении вернулся с охоты молодой вождь дикарей, умеет ли читать дочка отшельника, есть ли поблизости скалы, где можно укрыться от песчаной бури.
\paragraph{\textit{Как правило, проверки Неприятностей инициирует мастер. Ему же придется судить о том, насколько проверка вообще необходима в контексте сцены. Разумеется, если у мастера и игроков уже готовы ответы на все вопросы, проверка Неприятностей вряд ли будет использоваться слишком часто. Но даже в таких случаях не стоит полностью исключать ее из игры.}}