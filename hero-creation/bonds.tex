\section{УЗЫ}
\paragraph{Узы:} Узы представляют собой утверждения, которыми должен руководствоваться игрок, выбирая линию поведения героя. Это внешние и внутренние обстоятельства, мировоззрение и привычки, ощутимо влияющие на решения героя и определяющие выбор при прочих равных.
\paragraph{Число Уз:} игрок может выбрать для своего героя 0—2 Уз самостоятельно или определить их случайным образом. Когда герой связан Узами, протяните к герою 1 дополнительную Нить Судьбы в начале игровой встречи или сюжетной вехи за каждые Узы героя.
\paragraph{Темные Нити:} если герой совершает действия, противоречащие выбранным Узам, игрок должен оборвать 1 Нить или передать в руки мастера 1 Темную Нить.
\newline
Мастер может использовать Темные Нити в отношении статистов и персон так же, как используются Нити Судьбы в отношении героев, но только в тех случаях, когда статист или персона противостоят героям. Это единственный случай, когда статист может применить Ход Судьбы без проверок и последствий для себя!
\newline
Мастер может обрывать Темные Нити для ввода в игру Капризов Судьбы более 1 раза за сцену. Тем не менее, мастер все еще не может вводить один и тот же Недостаток, Темную сторону и Решку героя больше 1 раза за сцену. Например, если в распоряжении мастера есть 2 Темных Нити, он может 1 раз столкнуть героя с последствиями Недостатка бесплатно, 1 раз ввести в игру Темную сторону его Атрибута и 1 раз ввести в игру Решку героя, но не может ввести в игру один и тот же Недостаток героя дважды. В распоряжении мастера одновременно может находиться число Темных Нитей, равное \textbf{|1 + число игроков|}.
\paragraph{Смена Уз и отказ от них:} игрок может заменить одни Узы на другие или освободить героя от Уз, если это обусловлено логикой развития истории. Не считая специально оговоренных случаев, смена и отказ от Уз происходит в начале игровой встречи.
\paragraph{Выбор Уз и контекст:} Узы очень зависимы от логики происходящего. Если по какой-то причине в течение игровой встречи герой не может даже теоретически столкнуться с затруднениями и ограничениями, вызванными Узами, мастер начинает встречу с 1 Темной Нитью за каждые Узы, которые не будут задействованы.
\newline
Выбор Уз — это не только дополнительная Нить, но и способ сделать характер героя более выпуклым. Если вы задумаетесь над тем, почему герой связан именно этими Узами, то узнаете множество любопытных фактов о его личности. Например, если герой связан Узами единства, он начинает игру с 1 дополнительной Нитью, но должен оборвать 1 Нить или передать мастеру 1 Темную Нить, если его действия могут явно навредить героям остальных игроков. Почему командная работа так важна для него? Быть может, герой — ветеран, знающий, к чему может привести разлад на поле битвы, или в его большой семье все привыкли помогать друг другу. Не исключено даже, что герой — участник преступного синдиката, могущество которого держится на круговой поруке.
\newline
Чтобы определить Узы случайным образом, выберите столбец и бросьте К20.
\begin{center}
\begin{tabular}{ |c|p{7cm}|c|p{7cm}| }
\hline
K20 & \textbf{Узы} & K20 & \textbf{Узы} \\ \hline
1 & Узы единства & 1 & Лишний шум - лишние проблемы \\ \hline
2 & Любовь выдумали поэты и менестрели & 2 & Первый удар - решающий \\ \hline
3 & Честь превыше всего & 3 & Семь раз отмерь, один раз отрежь \\ \hline
4 & Сумел нынче убежать - завтра будешь воевать & 4 & Насилием нельзя изменить мир - только изуродовать \\ \hline
5 & Принимай чужеземцев с радушием & 5 & Слово - лучшее оружие \\ \hline
6 & После нас - хоть потоп & 6 & Цель оправдывает средства \\ \hline
7 & Простолюдины хуже зверей & 7 & Один раз живем \\ \hline
8 & Умеренность и аккуратность & 8 & Закон не ошибается \\ \hline
9 & Вещи не предадут и не обманут & 9 & Знание - сила \\ \hline
10 & Счастья за деньги не купишь & 10 & Старшим виднее \\ \hline
11 & Сделал дело - гуляй смело & 11 & Женщина священна \\ \hline
12 & От стариков никакой пользы & 12 & От чужаков добра не жди \\ \hline
13 & Каждая жизнь бесценна & 13 & Высокий род не оправдывает высокомерия \\ \hline
14 & Сомнения - удел слабаков & 14 & Жестокость внушает уважение \\ \hline
15 & Терпение украшает & 15 & Любовь - величайшее из чудес \\ \hline
16 & Ты - мне, я - тебе & 16 & Предательство постыдно \\ \hline
17 & Мужчина во всем главный & 17 & Никогда не сдавайся \\ \hline
18 & Око за око & 18 & Деньги могут все \\ \hline
19 & Приметы не лгут & 19 & Вовремя предать - значит предвидеть \\ \hline
20 & Я заслуживаю самого лучшего & 20 & Вещи обременяют \\ \hline
\end{tabular}
\end{center}
\paragraph{Узы единства} объясняют, почему герои, порой очень поверхностно знакомые друг с другом, работают в команде и не выступают друг против друга открыто или тайно, даже когда есть возможность извлечь из предательства серьезную выгоду. Узы единства не обязаны связывать всех героев в команде. Иногда один герой искренне радеет за общее дело, а другой ждет удобного момента, чтобы перерезать ему глотку! С другой стороны, в не нацеленной на внутренние конфликты команде Узы единства — хороший способ начать игру с дополнительной Нитью. Узы единства также распространяются и на статистов, которые важны для успеха дела.
\begin{center}
\begin{tabular}{ |c|p{4cm}|p{10cm}| }
\hline
\textbf{К20} & \textbf{Узы единства} & \textbf{Мотив} \\ \hline
1 & Один в поле не воин & Боязнь остаться один на один с некоей угрозой удерживает героя от необдуманных действий \\ \hline
2 & С приказами не спорят & Герой получил приказ, запрещающий ему вредить остальнму, по крайней мере, пока дело не завершено \\ \hline
3 & Лучше делить на всех, чем лишиться всего & Герой не уверен в своих силах и предпочитает получить меньше, но наверняка \\ \hline
4 & Дружба нерушима & Герой испытывает искреннюю симпатию к своим спутникам и считает их друзьями \\ \hline
5 & Без меня они пропадут & Герой чувствует ответственность за своих спутников, хоть и относится к ним слегка снисходительно \\ \hline
6 & Сначала дело, потом - разборки & Герой привык разделять деловые интересы и личную неприязнь. Это не мешает ему считать своих спутников ничтожествами, хотя говорить об этом вслух он вряд ли сочтет разумным \\ \hline
7 & Свары - для любителей & Герой считает себя профессионалом и не позволяет эмоциям взять верх над здравым смыслом. Впрочем, он не будет хвататься за оружие, если кто-то из его спутников действительно \textit{заслужил} хорошую взбучку \\ \hline
8 & Боги ненавидят предателей & Герой убежден, что боги покарают его за предательство \\ \hline
9 & Риск слишком велик & Герой рад бы обогатиться за чужой счет, но боится огласки и преследования властей \\ \hline
10 & Я выше этого & Герой не видет смысла в усобицах. Возможно, он в тайне гордится этим \\ \hline
11 & Для дела важен каждый & Герой уверен, что натянутые отношения с кем-то из спутников (не говоря уж о гибели), серьезно понизят общие шансы на успех \\ \hline
12 & За смирение мне воздастся & Герой верит, что Судьба воздаст ему за терпение по отношению к спутникам, как и за нежелание идти против них \\ \hline
13 & Ценные связи на будущее & Герой стремится сохранить со спутниками хорошие отношения из соображений будущей выгоды \\ \hline
14 & Я здесь не ради выгоды & Успех предприятия на первом месте для героя. Он скорее пожертвует собственной выгодой, чем пойдет на конфликт \\ \hline
15 & Доверие - основа успеха & Герой не представляет командной работы без взаимопомощи. Он полностью доверяет спутникам и ждет от них того же \\ \hline
16 & Честь дороже выгоды & Герой считает предательство бесчестным делом \\ \hline
17 & Я зла не делаю и не помню & Герой редко раздражается и быстро отходит. Причинить спутнику зло, пусть и ради выгоды, кажется ему чудовищным \\ \hline
18 & Просто немыслимо & Герой никогда не задумывался о мести и предательстве в принципе \\ \hline
19 & Мы - команда & Герой считает себя частью единого целого, команды, внутри которой каждый занимается своим делом и получает то, что должно \\ \hline
20 & Я - пример для окружающих & Герой считает себя примером добродетели и печется о своей репутации \\ \hline
\end{tabular}
\end{center}