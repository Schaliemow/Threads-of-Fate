\section{ОСНОВНЫЕ НАВЫКИ}
Значения Навыков отображают глубину познаний героя в различных областях. В скобках указана Характеристика, модификатор которой чаще всего прибавляется к Навыку при проверках, но могут возникнуть ситуации, при которых на для успеха приходится использовать другие характеристики, например, обычно для того, чтобы хорошо спрятаться нужно использовать навык Скрытность(Лв,Ин), но если герой хочет скрыться под водой, задержав дыхание или спрятаться в углу потолка, держась за стены, будет уместно использовать Скрытность(Вн). К некоторым Навыкам по умолчанию могут прибавляться модификаторы различных Характеристик, в зависимости от области применения. Мастер указывает перед проверкой, какой именно модификатор используется.
\paragraph{Значение навыка (Нв)} равно числу Очков опыта, распределенных в Навык.
\paragraph{Максимальное число Очков опыта в Навыке} не может превышать значение(не модификатор) Интеллекта героя. Когда герой совершает проверку Навыка, игрок бросает К20 и прибавляет к выпавшему результату значение Навыка героя и модификатор Характеристики, связанной с Навыком.
\newline
Если вы не используете правила Состязания, но герою кто-то активно противостоит, проверки Навыков совершаются следующим образом:
\begin{enumerate}
\item Бросьте К20 и прибавьте к нему значение Навыка и модификатор Характеристики.
\item Сравните получившееся число с \textbf{|Навыком оппонента +10|}. Герой преуспевает, если число равно сложности проверки или превышает ее.
\end{enumerate}
Например, герой со Скрытностью 6 и Ловкостью 14 (модификатор +2) пытается прокрасться мимо охранника. Охранник обладает Наблюдательностью 5 и Мудростью 12 (модификатор +1). Значит, для того, чтобы обмануть его бдительность, герою потребуется совершить проверку Скрытности против \textbf{|5 +1 +10 = 16|}. Если герой выбросит на К20 8 и больше, проверка будет успешна, т.к. \textbf{|6 +2 +8 = 16|}. В сумерках избежать внимания охранника будет гораздо легче — герой получит Преимущество. Напротив, если охранник бдительно осматривает пустой узкий коридор, герою придется действовать с Помехой, а то и двумя, если коридор ярко освещен!
\paragraph{Альтернативное применение Навыков:} использование Навыков — творческий процесс. Зачастую, добиться желаемого можно множеством разных способов, особенно, если к этому располагает контекст.
\paragraph{Нулевой уровень навыков:} если игрок не распределил в Навык героя хотя бы 1 Очко опыта, проверка этого Навыка совершается с Помехой.
\paragraph{Нулевой уровень боевых навыков:} если игрок не распределил хотя бы 1 Очко Опыта во Владение оружием, Рукопашный бой или Стрельбу героя, соответствующие проверки Доблести и Меткости совершаются с Помехой.

\genAndGet{basicSkills}
%\input{|python3 scripts/basicSkills.py}
%\input{scripts/output/basicSkills}
