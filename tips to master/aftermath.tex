\section{ПОСЛЕ ИГРЫ}
Игровая встреча завершается раздачей Очков опыта. За что его выдавать — часть предыгровой договоренности. Среднее количество Очков опыта, которые приобретают герои за время одной игровой встречи, варьируется от 1 до 5. В это число не входят Очки опыта, которые герой получил, принимая влияние других героев или статистов.
\newline
Поскольку «Нити Судьбы» — ролевая игра, при начислении опыта вы можете (хоть и не обязаны) учитывать как успехи героев в решении внутриигровых проблем, так и игру роли в исполнении игроков.
\newline
Разумеется, вы можете выделить особо удачные идеи кого-то из игроков и начислить их героям больше опыта, но лучше этим не злоупотреблять. В конце концов, главная награда в настольной ролевой игре — сам процесс и общение с единомышленниками.